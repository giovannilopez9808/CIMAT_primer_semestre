\subsection{Método de las potencias \label{sec:power}}

Se define un vector inicial $v_0$, el cual será la inicialización del método.
Los elementos del vector $v_0$ son iguales a

\begin{equation*}
    v_{0,i} = \frac{1}{\sqrt{n}}
\end{equation*}

Con esta definición considerá que el vector $v_0$ debe ser siempre normalizado. El método aplica la ecuación \ref{eq:power_iterative}, conforme más se itera el método ira convergiendo hacia el eigenvector de $\lambda_1$.

\begin{equation}
    v_i = Av_{i-1} \label{eq:power_iterative}
\end{equation}

Despues de aplicar la ecuación \ref{eq:power_iterative}, el vector $v_i$ deberá ser normalizado antes de volverse a aplicar al método. Para obtener el eigenvalor $\lambda_1$, se usa la siguiente ecuación \ref{eq:power_lambda_i}.

\begin{equation}
    \lambda_i = \frac{\langle v_i , v_{i-1}\rangle}{\langle v_{i-1}, v_{i-1}\rangle} \label{eq:power_lambda_i}
\end{equation}

La convegencia del algoritmo que se implemento fue usando la ecuación \ref{eq:power_tol}.

\begin{equation}
    \theta = |\lambda_i - \lambda_{i-1}| \label{eq:power_tol}
\end{equation}

Entonces, el método de las potencias es descrito en el pseudo código \ref{alg:power_method}.

\begin{algorithm}[H]
    \caption{Método de las potencias}
    \label{alg:power_method}
    \KwInput{$A$}
    \KwOutput{$v_i$ y $\lambda_i$}
    \While{$\theta > 10^{-6}$}
    {
        $v_i \gets Av_{i-1}$\\
        $\lambda_i \gets \frac{\langle v_i , v_{i-1}\rangle}{\langle v_{i-1}, v_{i-1}\rangle}$\\
        $v_i \gets normalize(v_i)$\\
    }
\end{algorithm}