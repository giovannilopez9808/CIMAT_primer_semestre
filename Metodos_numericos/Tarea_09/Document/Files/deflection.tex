\subsection{Método de deflación}

Usando el método de potencias descrito en la sección \ref{sec:power} podemos tener una aproximación del eigenvalor dominante y su eigenvector asociado. El método de deflación calcula los m eigenvalor de una matriz, este se define de la siguiente manera:

Sean $\lambda_1,\lambda_2,\dots,\lambda_n$ los valores propios de la matriz A. Supongamos que $\lambda_1$ es el eigenvalor dominante, $v_1$ su eigenvector asociado Y $v$ un vector tal que $\langle v_1, v \rangle = 1$. Sea B la matriz definida como

\begin{equation*}
    B =A -\lambda_1 v_1 v^T
\end{equation*}

entonces los valores propios de la matriz B son $0,\lambda_2,\lambda_3,\dots,\lambda_n$.

Entonces al haber `eliminado' al eigenvalor dominante de la matriz, al aplicar el método de potencias encontraremos un eigenvalor distinto. Tomaremos a $v=v_1$, esto porque al ser $v_1$ normalizado, entonces su producto punto consigo mismo debe ser 1.

El método de deflación es descrito en el pseudo código \ref{alg:deflation_method}.


\begin{algorithm}[H]
    \caption{Método de deflación}
    \label{alg:deflation_method}
    \KwInput{$A,m$}
    \KwOutput{$v$ y $\lambda$}
    \For{i=1,m}{
        $v_0 \gets initialize\_vector(v_0)$ \\
        \For{j=1,i-1}{
            $v_0\gets v_0 - \langle v_{j}, v_{j}\rangle v_{j}$\\
        }
        \While{$\theta > 10^{-6}$}
        {
            $v_i \gets Av_{i-1}$\\
            $\lambda_i \gets \frac{\langle v_i , v_{i-1}\rangle}{\langle v_{i-1}, v_{i-1}\rangle}$\\
            $v_j \gets normalize(v_j)$\\
            \For{j=1,i-1}{
                $v_i\gets v_i - \langle v_{j}, v_{j}\rangle v_{j}$\\
            }
        }
    }
\end{algorithm}
