\subsection{Método de Jacobi}

La Compilación de los programas se realizo con el siguiente comando:

\begin{lstlisting}[language=bash]
    gcc -Wall -Wextra -Werror -pedantic -ansi -o main.out main.c -std=c11 -lm
\end{lstlisting}

La ejecución del programa se realiza con el siguiente comando:

\begin{lstlisting}[language=bash]
    ./main.out matrix vector
\end{lstlisting}

donde matrix es el nombre del archivo de la matriz y vector es el nombre del archivo del vector de términos independientes. Por ejemplo, para ejecutar el programa con los archivos \file{M\_sys\_3x3.txt} y \file{V\_sys\_\-3x1.txt} el comando tendria que ser el siguiente:

\begin{lstlisting}[language=bash]
    ./main.out M_sys_3x3.txt V_sys_3x1.txt
\end{lstlisting}