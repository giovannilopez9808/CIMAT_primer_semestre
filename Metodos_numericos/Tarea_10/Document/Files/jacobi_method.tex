\subsection{Método de Jacobi}

El método de Jacobi recibe una matriz simetrica A, la cual reducirá a una matriz diagonal por medio de transformaciones dadas por matrices ortogonales. Consideremos la matriz de rotación $J(i,j,\theta)$ (ecuación \ref{eq:matrix_jacobi}).

\begin{equation}
    J(i,j,\theta)=\begin{pmatrix}
        1      & \cdots & 0            & \cdots & 0           & \cdots & 0      \\
        \vdots & \ddots & \vdots       &        & \vdots      &        & \vdots \\
        0      & \cdots & cos(\theta)  & \cdots & sin(\theta) & \cdots & 0      \\
        \vdots &        & \vdots       & \ddots & \vdots      &        & \vdots \\
        0      & \cdots & -sin(\theta) & \cdots & cos(\theta) & \cdots & 0      \\
        \vdots &        & \vdots       &        & \vdots      & \ddots & \vdots \\
        0      & \cdots & 0            & \cdots & 0           & \cdots & 1      \\
    \end{pmatrix}
    \label{eq:matrix_jacobi}
\end{equation}

donde $i,j$ denotan la posición de los elementos diferentes en la matriz identidad. Los valores de la matriz $J$, deben satisfacer la siguiente relación:

\begin{equation*}
    \begin{pmatrix}
        c  & s \\
        -s & c
    \end{pmatrix}^T
    \begin{pmatrix}
        a_{ii} & a_{ij} \\
        a_{ij} & a_{jj}
    \end{pmatrix}
    \begin{pmatrix}
        c  & s \\
        -s & c
    \end{pmatrix} =
    \begin{pmatrix}
        a'_{ii} & 0       \\
        0       & a'_{jj}
    \end{pmatrix}^T
\end{equation*}

donde $c=cos(\theta)$ y $s=sin(\theta)$. Entonces, se puede deducir que:

\begin{equation*}
    a_{ij}(c^2-s^2) + (a_{ii}-a_{jj})cs = 0
\end{equation*}

definimos como:

\begin{equation*}
    \tau = \frac{a_{jj}-a_{ii}}{2a_{ij}} ; \qquad t = \frac{s}{c}
\end{equation*}

donde t debe satisfacer a la ecuación cuadratica

\begin{equation*}
    t^2 - 2\tau t -1 =0
\end{equation*}

por lo tanto

\begin{equation*}
    c = \frac{1}{\sqrt{1+t^2}} \qquad s =ct
\end{equation*}

La manera en que se elijen las posiciones $i,j$ es encontrando la posición del elemento con valor absoluto mayor de la matriz $A$ fuera de la diagonal.

La convergencia de este método es dado con la siguiente ecuación:

\begin{equation*}
    max\{a_{ij}\} < tol \qquad i\neq j
\end{equation*}

Los eigenvalores se encontrarán en la diagonal de la matriz A al final del método. Los eigenvectores asociados a cada eigenvalor pueden ser calculados con la siguiente operación

\begin{equation*}
    v = J_1J_2J_3\dots J_n
\end{equation*}

Por lo tanto, el algoritmo de Jacobi puede ser escrito de la siguiente manera:

\begin{algorithm}[H]
    \caption{Método de Jacobi}
    \label{alg:jacobi_method}
    \KwInput{$A$}
    \KwOutput{$v$ y $\lambda$}
    $v \gets I_{n\times n}$\\
    \While{$|max\{ a_{ij}\}|>tol$}{
        $J \gets obtain\_jacobi\_elements(A)$ \\
        $A \gets J^TAJ$\\
        $v \gets vJ$
    }
\end{algorithm}
