\subsection{Método de Jacobi}

El método de Jacobi recibe una matriz simetrica A, la cual reducirá a una matriz diagonal por medio de transformaciones dadas por matrices ortogonales. Consideremos la matriz de rotación $J(i,j,\theta)$ (ecuación \ref{eq:matrix_jacobi}).

\begin{equation}
    J(i,j,\theta)=\begin{pmatrix}
        1      & \cdots & 0            & \cdots & 0           & \cdots & 0      \\
        \vdots & \ddots & \vdots       &        & \vdots      &        & \vdots \\
        0      & \cdots & cos(\theta)  & \cdots & sin(\theta) & \cdots & 0      \\
        \vdots &        & \vdots       & \ddots & \vdots      &        & \vdots \\
        0      & \cdots & -sin(\theta) & \cdots & cos(\theta) & \cdots & 0      \\
        \vdots &        & \vdots       &        & \vdots      & \ddots & \vdots \\
        0      & \cdots & 0            & \cdots & 0           & \cdots & 1      \\
    \end{pmatrix}
    \label{eq:matrix_jacobi}
\end{equation}

donde $i,j$ denotan la posición de los elementos diferentes en la matriz identidad. Los valores de la matriz $J$, deben satisfacer la siguiente relación:

\begin{equation*}
    \begin{pmatrix}
        c  & s \\
        -s & c
    \end{pmatrix}^T
    \begin{pmatrix}
        a_{ii} & a_{ij} \\
        a_{ij} & a_{jj}
    \end{pmatrix}
    \begin{pmatrix}
        c  & s \\
        -s & c
    \end{pmatrix} =
    \begin{pmatrix}
        a'_{ii} & 0       \\
        0       & a'_{jj}
    \end{pmatrix}^T
\end{equation*}

donde $c=cos(\theta)$ y $s=sin(\theta)$. Entonces, se puede deducir que:

\begin{equation*}
    cos(2\theta)a_{ij}+\frac{1}{2}sin(2\theta)(a_{ii}-a_{jj})=0
\end{equation*}

entonces:

\begin{equation*}
    \theta = \frac{1}{2}arctan\left(\frac{2a_{ij}}{a_{ii}-a_{jj}}\right)
\end{equation*}

en el caso en que $a_{ii}=a_{jj}$, entonces $\theta=\frac{\pi}{4}$.

La manera en que se elijen las posiciones $i,j$ es encontrando la posición del elemento con valor absoluto mayor de la matriz $A$ fuera de la diagonal.

La convergencia de este método es dado con la siguiente ecuación:

\begin{equation*}
    max\{a_{ij}\} < tol \qquad i\neq j
\end{equation*}


Se busca esta convergencia en los valores de  $i=1,2,\dots,n-1$ $j=i+1,i+2,\dots,n$. Esto es porque la matriz A se supone que es simetrica.

Los eigenvalores se encontrarán en la diagonal de la matriz A al final del método. Los eigenvectores asociados a cada eigenvalor pueden ser calculados con la siguiente operación

\begin{equation*}
    v = J_1J_2J_3\dots J_n
\end{equation*}

Por lo tanto, el algoritmo de Jacobi puede ser escrito de la siguiente manera:

\begin{algorithm}[H]
    \caption{Método de Jacobi}
    \label{alg:jacobi_method}
    \KwInput{$A$}
    \KwOutput{$v$ y $\lambda$}
    $v \gets I_{n\times n}$\\
    \While{$|max\{ a_{ij}\}|>tol$}{
        $i,j \gets pos(|max\{a_{ij}\}|)$\\
        $\theta = \frac{1}{2}arctan\left(\frac{2a_{ij}}{a_{ii}-a_{jj}}\right)$\\
        $A \gets J^TAJ$\\
        $v \gets vJ$
    }
\end{algorithm}

En este trabajo, el proceso de la linea 5 y 6 se optimizo para realizar unicamente la multiplicación en los elementos afectados por la rotación. Usando las siguentes relaciones:

\begin{align}
    a^{(k)}_{ip} & = a^{(k-1)}_{ip} cos(\theta) + a^{(k)}_{iq} sin(\theta) \label{eq:equation1} \\
    a^{(k)}_{iq} & = a^{(k)}_{iq} cos(\theta) - a^{(k-1)}_{ip} sin(\theta) \label{eq:equation2} \\
    a^{(k)}_{pj} & = a^{(k-1)}_{pj} cos(\theta) + a^{(k)}_{qj} sin(\theta)                      \\
    a^{(k)}_{qj} & = a^{(k)}_{qj} cos(\theta) - a^{(k-1)}_{pj} sin(\theta)
\end{align}

donde $a^{(0)}_{ij}$ son los elementos de la matriz A. Para el calculo de los eigenvectores se usan las ecuaciones \ref{eq:equation1} y \ref{eq:equation2} tomando en cuenta que $a^{0} = I$.

