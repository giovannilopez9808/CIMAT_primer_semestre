\section{Conclusiones}

El método de la potencia inversa con deflación tiende a realizar tener un tiempo mayor de calculo. En nuestro caso para la matriz de dimension 1000x1000 esta se tardo 20 horas en tener el resultado. La ventaja de este método es la definición de cuantos eigenvalores queremos calcular de la matriz dada.

El método de Jacobi tiende a realizar un buen trabajo con matrices simetricas, ya que para esto esta diseñado, este método puede llegar a tener un tiempo de ejecucción más largo. Esto es si seguimos la definición del método, pero si se realiza una optimización intermedia al realizar la multiplicación de matrices, este método es más eficiente. El método de Jacobi tardo 8 horas en llegar al resultado de la matriz de dimension 1000x1000.