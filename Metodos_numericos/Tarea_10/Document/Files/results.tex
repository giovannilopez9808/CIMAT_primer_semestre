\section{Resultados}

Para la demostración de algunos resultados se usaron las siguientes matrices:

\begin{equation}
    A = \begin{pmatrix}
        3    & -0.1 & -0.2 \\
        -0.1 & 7    & -0.3 \\
        -0.2 & -0.3 & 10
    \end{pmatrix}
    \qquad
    B = \begin{pmatrix}
        1 & 0 & 2 \\
        0 & 2 & 1 \\
        2 & 1 & 1
    \end{pmatrix}
\end{equation}
\subsection{Método de la potencia inversa con deflación}

Para el método de la potencia inversa con deflación se obtuvieron los siguientes eigenvalores para las matrices A y B.

\begin{equation*}
    \lambda_A = \begin{pmatrix}
        2.991343 \\
        6.973860
    \end{pmatrix}
    \qquad
    \lambda_B = \begin{pmatrix}
        -1.164248 \\
        1.772866
    \end{pmatrix}
\end{equation*}

con sus respectivos eigenvectores:

\begin{equation*}
    v_A = \begin{pmatrix}
        0.999190 & -0.029942 \\
        0.027185 & 0.994855  \\
        0.029679 & 0.096782
    \end{pmatrix}
    \qquad
    v_B = \begin{pmatrix}
        0.661072  & -0.497285 \\
        0.226165  & 0.846066  \\
        -0.715425 & -0.192040
    \end{pmatrix}
\end{equation*}


\subsection{Método de Jacobi}

Para el método de Jacobi se obtuvieron los siguientes eigenvalores para las matrices A y B.

\begin{equation*}
    \lambda_A = \begin{pmatrix}
        2.991343 \\
        6.973860 \\
        10.034796
    \end{pmatrix}
    \qquad
    \lambda_B = \begin{pmatrix}
        3.391382 \\
        1.772866 \\
        -1.164248
    \end{pmatrix}
\end{equation*}

Los eigenvectores correspondientes a cada eigenvalor son los siguientes:

\begin{equation*}
    v_A = \begin{pmatrix}
        0.999191 & -0.029900 & -0.026899 \\
        0.027147 & 0.994869  & -0.097461 \\
        0.029675 & 0.096651  & 0.994876
    \end{pmatrix}
    \qquad
    v_B = \begin{pmatrix}
        0.561818 & -0.497279 & -0.661115 \\
        0.482801 & 0.846041  & -0.226091 \\
        0.671761 & -0.192165 & 0.715409
    \end{pmatrix}
\end{equation*}