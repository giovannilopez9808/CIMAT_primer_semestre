\subsection{Eliminación Gaussiana}

Suponiendo que se tiene un sistema de ecuaciones de  n ecuaciones con n coeficientes, donde cada coeficiente es un número real. Entonces representando el sistema de ecuaciones como una ecuación matricial se obtiene lo siguiente:

\begin{equation*}
    \begin{pmatrix}
        a_{11} & a_{12} & \cdots & a_{1n} \\
        a_{21} & a_{22} & \cdots & a_{2n} \\
        \vdots & \vdots & \ddots & \vdots \\
        a_{n1} & a_{n2} & \cdots & a_{nn}
    \end{pmatrix}
    \begin{pmatrix}
        x_1    \\
        x_2    \\
        \vdots \\
        x_n
    \end{pmatrix} =
    \begin{pmatrix}
        b_1    \\
        b_2    \\
        \vdots \\
        b_n
    \end{pmatrix}
\end{equation*}

Llamaremos a la matriz expandida aquella matriz de tamaño n$\times$n+1, tal que:

\begin{equation}
    \left(\begin{matrix}
            a_{11} & a_{12} & \cdots & a_{1n} \\
            a_{21} & a_{22} & \cdots & a_{2n} \\
            \vdots & \vdots & \ddots & \vdots \\
            a_{n1} & a_{n2} & \cdots & a_{nn}
        \end{matrix}\right| \left.
    \begin{matrix}
            b_1    \\
            b_2    \\
            \vdots \\
            b_n
        \end{matrix}
    \right)
    \label{eq:extended_matrix}
\end{equation}

Se realizaran operaciones entre renglones de la matriz extendida (\ref{eq:extended_matrix}) de tal manera que se llegue a una matriz triangular superior. Se obtendra un término que llamaremos $r_ij$, el cual esta definido como en la ecuación \ref{eq:def_ratio}.

\begin{equation}
    r_{ji} = \frac{a_{ii}}{a_{ji}}
    \label{eq:def_ratio}
\end{equation}

Donde la posición j, es el número de fila que será restada por la fila i. Por ejemplo, para iniciar se tomara a $i=1$, y se restara la fila 2 ($j=2$), esto para obtener que el elemento $a_{21}$ sea igual a 0. Aplicando este proceso a la matriz \ref{eq:extended_matrix} se obtiene el siguiente resultado:

\begin{equation*}
    \left(\begin{matrix}
        a_{11}       & a_{12}       & \cdots & a_{1n}       \\
        a_{21}r_{21} & a_{22}r_{21} & \cdots & a_{2n}r_{21} \\
        \vdots       & \vdots       & \ddots & \vdots       \\
        a_{n1}       & a_{n2}       & \cdots & a_{nn}
    \end{matrix}\right| \left.
    \begin{matrix}
        b_1r_{21} \\
        b_2       \\
        \vdots    \\
        b_n
    \end{matrix}
    \right)
\end{equation*}

Restando la linea 2 con la linea 1 se obtiene lo siguiente:

\begin{equation*}
    \left(\begin{matrix}
        a_{11}              & a_{12}              & \cdots & a_{1n}              \\
        a_{21}r_{21}-a_{11} & a_{22}r_{21}-a_{12} & \cdots & a_{2n}r_{21}-a_{1n} \\
        \vdots              & \vdots              & \ddots & \vdots              \\
        a_{n1}              & a_{n2}              & \cdots & a_{nn}
    \end{matrix}\right| \left.
    \begin{matrix}
        b_1       \\
        b_2r_{21} \\
        \vdots    \\
        b_n
    \end{matrix}
    \right)
\end{equation*}

Donde el termino $a_{21}r_{21}-a_{11}=0$, renombrando a cada termino de la fila 2, como $a_{2i}^*$, se obtiene que:

\begin{equation*}
    \left(\begin{matrix}
        a_{11} & a_{12}   & \cdots & a_{1n}   \\
        0      & a_{22}^* & \cdots & a_{2n}^* \\
        \vdots & \vdots   & \ddots & \vdots   \\
        a_{n1} & a_{n2}   & \cdots & a_{nn}
    \end{matrix}\right| \left.
    \begin{matrix}
        b_1    \\
        b_2^*  \\
        \vdots \\
        b_n
    \end{matrix}
    \right)
\end{equation*}

repitiendo este proceso con las demás filas se llega a la siguiente matriz:

\begin{equation*}
    \left(\begin{matrix}
        a_{11} & a_{12}   & \cdots & a_{1n}   \\
        0      & a_{22}^* & \cdots & a_{2n}^* \\
        \vdots & \vdots   & \ddots & \vdots   \\
        0      & a_{n2}^* & \cdots & a_{nn}^*
    \end{matrix}\right| \left.
    \begin{matrix}
        b_1    \\
        b_2^*  \\
        \vdots \\
        b_n^*
    \end{matrix}
    \right)
\end{equation*}

El proceso se repetira tomando el valor de cada diagonal de la siguiente fila y restando las filas que estan por debajo de esta, para asi obtener una matriz triangular superior y aplicar el método de solución antes descrito para este sistema de ecuaciones.

El algoritmo planteado para realizar la solución a este tipo de sistemas de ecuaciones es el siguiente:

\begin{lstlisting}[style=CStyle]
    // inputs: matriz, vector_b
    // output: solutions
    for(i = 1; i <= n-1; i++)
    {
        a_ii = matriz[i][i]
        b_i = vector_b[i]
        for(j = i + 1; j <= n; j++)
        {
            a_ij = matriz[i][j]
            ratio_ji = a_ii / a_ij
            b_j = b_j * ratio_ji - b_i
            for(k = i; k < n,k++)
            {
                matriz_ik = matriz[i][k]
                matriz_jk = matriz_jk * ratio_ji - matriz_ik
            }
        }
    }
    solve_triangular_superior_matrix(matrix,vector_b)
\end{lstlisting}

La imprementación de este problema se encuentra en la carpeta \folder{Problema\_4}. La función \script{solve\_gaussian\_matrix} esta contenido en el archivo \file{solution.h}