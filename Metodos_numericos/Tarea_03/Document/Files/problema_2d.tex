Calculando la diferecia relativa entre los datos obtenidos se obtienen los resultados que se muestran en la tabla \ref{table:results2d}.

\begin{table}[H]
      \centering
      \begin{tabular}{lcc}
            \hline
            \textbf{Función}                                   & \textbf{f(x)} & \textbf{\begin{tabular}[c]{@{}l@{}}Diferencia\\ Relativa\end{tabular}} \\ \hline
            Análitica (Ecuación \ref{eq:problem2fx})           & -1.998999     & -                                  \\
            Redondeo (Ecaución \ref{eq:problema2b})            & -1.500000     & 24.9625                            \\
            Serie de potencias (Ecuación \ref{eq:problema_fc}) & -1.500000     & 24.9625                            \\ \hline
      \end{tabular}
      \caption{Valores obtenidos en los ejercicos 2b y 2c comparandolos con el resultado análitico de la ecuación \ref{eq:problem2fx} en x=0.1.}
      \label{table:results2d}
\end{table}
