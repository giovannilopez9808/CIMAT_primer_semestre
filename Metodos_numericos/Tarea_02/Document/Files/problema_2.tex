\section*{Problema 2}
\textbf{Implement a function to comput
    \begin{equation}
        f(x)=\frac{1}{\sqrt{x^2+1}-x}
        \label{eq:ejercicio2}
    \end{equation}
    When evaluating the previous function we can lose accuracy, transform the right hand side to avoid error (or improve the accuracy). Implement the transformed expression and compare the results with the original function. Note: use values of x greater than 10000.}

La función \ref{eq:ejercicio2} puede provocar problemas con números grandes, ya que el termino $\sqrt{x^2+1}$ puede tener un valor muy cercano a $x$, provocando asi el obtener el resultado de 1/0. Racionalizando la función se obtiene lo siguiente:
\begin{align}
    f(x) & = \frac{1}{\sqrt{x^2+1}-x}                                                     \nonumber \\
         & = \frac{1}{\sqrt{x^2+1}-x} \left(\frac{\sqrt{x^2+1}+x}{\sqrt{x^2+1}+x} \right) \nonumber \\
         & = \frac{\sqrt{x^2+1}+x}{x^2+1-x^2}                                             \nonumber \\
         & = \frac{\sqrt{x^2+1}+x}{1}                                                     \nonumber \\
    f(x) & = \sqrt{x^2+1}+x \label{eq:ejercicio2_racionalizada}
\end{align}
Se crearon dos funciones donde se calculará $f(x)$ con las ecuaciones \ref{eq:ejercicio2} y \ref{eq:ejercicio2_racionalizada}, como entrada recibiran un número del tipo double y daran de salida un número de tipo double. Los resultados de cada función con diferentes valores de x se encuentran enlistados en la tabla \ref{table:problema2}.
\begin{table}[H]
    \centering
    \begin{tabular}{llll} \hline
        \textbf{x}      & \textbf{Ecuación \ref{eq:ejercicio2}} & \textbf{Ecuación \ref{eq:ejercicio2_racionalizada}} & \textbf{Diferencia} \\ \hline
        1.000000        & 2.414214                              & 2.414214                                            & 0.000000            \\
        10.000000       & 20.049876                             & 20.049876                                           & 0.000000            \\
        100.000000      & 200.005000                            & 200.005000                                          & 0.000000            \\
        1000.000000     & 2000.000500                           & 2000.000500                                         & 0.000000            \\
        10000.000000    & 19999.999778                          & 20000.000050                                        & 0.000272            \\
        100000.000000   & 200000.223331                         & 200000.000005                                       & 0.223326            \\
        1000000.000000  & 1999984.771129                        & 2000000.000001                                      & 15.228871           \\
        10000000.000000 & 19884107.851852                       & 20000000.000000                                     & 115892.148148       \\ \hline
    \end{tabular}
    \caption{Valores de x para las diferences funciones.}
    \label{table:problema2}
\end{table}

El programa de este problema se encuentra en la carpeta \textcolor{citecolor}{Problema\_2}. El comando para compilarlo es el siguiente:

\begin{lstlisting}[language=bash]
    gcc -Wall -o main.out main.c -lm -std=c11
\end{lstlisting}

\begin{lstlisting}[language=bash]
    gcc -Wall -o main.out main.c -lm -std=c11
\end{lstlisting}

El output esperado del programa es el siguiente:

\begin{lstlisting}[language=bash]
    ------------------------------
    Para x	= 1.000000
    f(x)	= 2.414214
    fr(x)	= 2.414214
    RD(x)	= 0.000000
    ------------------------------
    Para x	= 10.000000
    f(x)	= 20.049876
    fr(x)	= 20.049876
    RD(x)	= 0.000000
    ------------------------------
    Para x	= 100.000000
    f(x)	= 200.005000
    fr(x)	= 200.005000
    RD(x)	= 0.000000
    ------------------------------
    Para x	= 1000.000000
    f(x)	= 2000.000500
    fr(x)	= 2000.000500
    RD(x)	= 0.000000
    ------------------------------
    Para x	= 10000.000000
    f(x)	= 19999.999778
    fr(x)	= 20000.000050
    RD(x)	= 0.000272
    ------------------------------
    Para x	= 100000.000000
    f(x)	= 200000.223331
    fr(x)	= 200000.000005
    RD(x)	= 0.223326
    ------------------------------
    Para x	= 1000000.000000
    f(x)	= 1999984.771129
    fr(x)	= 2000000.000001
    RD(x)	= 15.228871
    ------------------------------
    Para x	= 10000000.000000
    f(x)	= 19884107.851852
    fr(x)	= 20000000.000000
    RD(x)	= 115892.148148
\end{lstlisting}
