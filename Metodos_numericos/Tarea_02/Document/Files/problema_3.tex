\section*{Problema 3}
\textbf{Implement a function to compute the exponential function by using the
    Taylor/Maclaurin series
    \begin{equation*}
        e^x= 1+ \frac{x}{1!}+\frac{x^2}{2!}+\frac{x^3}{3!}+ \cdots
    \end{equation*}
    Since we cannot add infinite terms, we can approximate this expansion by
    \begin{equation*}
        e^x= \sum_{i=0}^n \frac{x^i}{i!}
    \end{equation*}
}

Expandiendo hasta $n=3$, se obtiene que:
\begin{align*}
    f(x) & = a_0+a_1x+a_2x^2+a_3x^3 \\
         & = a_0+x(a_1+x(a_2+xa_3))
\end{align*}

Por lo que podemos obtener una ecuación recursiva de tal manera que:
\begin{align*}
    P_0 & = a_n                \\
    P_1 & = a_{n-1}+xP_0       \\
    P_2 & = a_{n-2}+xP_{1}     \\
        & \qquad \vdots        \\
    P_i & = a_{n-i} + xP_{i-1} \\
        & \qquad \vdots        \\
    P_n & = a_{0} +xP_{n-1}    \\
\end{align*}

El programa de este problema se encuentra en la carpeta \textcolor{citecolor}{Problema\_3}. El comando para compilarlo es el siguiente:

\begin{lstlisting}[language=bash]
    gcc -Wall -o main.out main.c -lm -std=c11
\end{lstlisting}

El output esperado del programa es el siguiente:

\textbf{Caso de datos de prueba}
\begin{lstlisting}[language=bash]
    Deseas usar el programa con los datos de prueba?(Y/n): y
    f(2.000000) = 7.388995
\end{lstlisting}

\textbf{Caso de datos de introducidos por usuario}
\begin{lstlisting}[language=bash]
    Deseas usar el programa con los datos de prueba?(Y/n): n
    Escribe el numero de terminos de la serie que quieres calcular:
    4
    Escribe el numero de x que quieres evaluar:
    -2
    f(-2.000000) = 0.333333    
\end{lstlisting}