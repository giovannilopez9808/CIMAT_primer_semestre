Calculando la diferecia relativa entre los datos obtenidos se obtienen los resultados mostrados en la tabla \ref{table:results2d}.

\begin{table}[H]
    \centering
    \begin{tabular}{lcc}
        \hline
        \textbf{Función}                                   & \textbf{f(x)} & \textbf{\begin{tabular}[c]{@{}l@{}}Diferencia\\ Relativa\end{tabular}} \\ \hline
        Análitica (Ecuación \ref{eq:problem2fx})           & 59.6306       & -                                  \\
        Redondeo (Ecaución \ref{eq:problema2b})            & 49.5000       & 16.9889                            \\
        Serie de potencias (Ecuación \ref{eq:problema_fc}) & 49.5000       & 16.9889                            \\ \hline
    \end{tabular}
    \caption{Valores obtenidos en los ejercicos 2b y 2c comparandolos con el resultado análitico de la ecuación \ref{eq:problem2fx} en x=0.1.}
    \label{table:results2d}
\end{table}