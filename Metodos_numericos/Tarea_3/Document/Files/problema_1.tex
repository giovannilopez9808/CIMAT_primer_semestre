\section*{Problema 1}
\textbf{Use four-digit rounding arithmetic and the quadratic formulas to find the
    most accurate approximations to the roots of the following quadratic equa-
    tions. Also use the form of the quadratic formula by rationalizing the
    numerator. Compute the absolute errors and relative errors.}
\begin{align}
    \frac{1}{3}x^2 - \frac{123}{4}x +\frac{1}{6} & = 0 \label{eq:funcion1} \\
    \frac{1}{3}x^2 + \frac{123}{4}x -\frac{1}{6} & = 0 \label{eq:funcion2}
\end{align}

La ecuación cuadrática es:
\begin{equation}
    x = \frac{-b\pm \sqrt{b^2-4ac}}{2a}
    \label{eq:problema1}
\end{equation}

Realizando el proceso de racionalización se obtiene que:
\begin{align}
    x & = \frac{-b\pm \sqrt{b^2-4ac}}{2a}\left(\frac{b\pm \sqrt{b^2-4ac}}{b\pm \sqrt{b^2-4ac}}\right) \nonumber \\
      & = \frac{-b^2+b^2-4ac}{2a(b\pm \sqrt{b^2-4ac})}                                   \nonumber              \\
    x & = \frac{-2c}{b\pm \sqrt{b^2-4ac}} \label{eq:problema1raiz}
\end{align}

Los resultados del programa se encuentran en la tabla \ref{table:resultados1}.
\begin{table}[H]
    \centering
    \begin{tabular}{llll} \hline
        \textbf{Problema}                           & \textbf{Función}                & \multicolumn{1}{c}{$\mathbf{x_1}$} & \multicolumn{1}{c}{$\mathbf{x_2}$} \\ \hline
        \multirow{2}{*}{Problema \ref{eq:funcion1}} & Ecuación \ref{eq:problema1}     & 0.0054                             & -92.2646                           \\
                                                    & Ecuación \ref{eq:problema1raiz} & 0.005420                           & -92.255420                         \\
        \multirow{2}{*}{Problema \ref{eq:funcion2}} & Ecuación \ref{eq:problema1}     & 0.0054                             & 92.2538                            \\
                                                    & Ecuación \ref{eq:problema1raiz} & 0.005420                           & 92.244580                          \\ \hline
    \end{tabular}
    \caption{Resultados de las raices de las ecuaciones \ref{eq:funcion1} y \ref{eq:funcion2} usando las ecuaciones \ref{eq:problema1} y \ref{eq:problema1raiz}.}
    \label{table:resultados1}
\end{table}

Para este caso se definieron la diferencia relativa y absoluta de la siguiente manera:
\begin{equation*}
    DA = \left|Y-X \right| \qquad DR = \frac{DA}{Y}*100
\end{equation*}
donde $Y$ son los resultados usando la ecuación \ref{eq:problema1raiz} y $X$ son los resultados de usar la ecuación \ref{eq:problema1}.
Obtiendo así, los calculos de la diferencia absoluta y diferencia relativa mostrados en la tabla \ref{table:difabsolute}.
\begin{table}[H]
    \centering
    \begin{tabular}{llll}
        \hline
        \multicolumn{1}{c}{\textbf{Problema}}       & \textbf{Valores} & \multicolumn{1}{c}{\textbf{\begin{tabular}[c]{@{}c@{}}Diferencia\\ absoluta\end{tabular}}} & \multicolumn{1}{c}{\textbf{\begin{tabular}[c]{@{}c@{}}Diferencia\\ Relativa\end{tabular}}}    \\ \hline
        \multirow{2}{*}{Problema \ref{eq:funcion1}} & x1               & 0.000020                                               & 0.369004                                               \% \\
                                                    & x2               & 0.009183                                               & 0.009954                                               \% \\
        \multirow{2}{*}{Problema \ref{eq:funcion2}} & x1               & 0.009220                                               & 0.009994                                               \% \\
                                                    & x2               & 0.000020                                               & 0.377269                                               \% \\ \hline
    \end{tabular}
    \caption{Diferencias absolutas y relativas de los resultados de la tabla \ref{table:resultados1}.}
    \label{table:difabsolute}
\end{table}

El programa se encuentra en la carpeta \textcolor{citecolor}{Problema\_1}. Para compilar el programa se debe ingresar la siguiente linea:
\begin{lstlisting}[language=bash]
    gcc -Wall -o main.out main.c -lm -std=c11    
\end{lstlisting}