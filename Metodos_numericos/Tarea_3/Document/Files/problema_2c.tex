Escribiendo las ecuaciones de seno y coseno como sereis de potenicas, se obtiene lo siguiente:
\begin{equation*}
    sin(x) = \sum_{i=0}^n \frac{(-1)^i}{(2i+1)!} x^{2i+1} \qquad
    cos(x) = \sum_{i=0}^n \frac{(-1)^i}{(2i)!} x^{2i} \qquad
\end{equation*}

Transformando estas ecuaciones a una ecuación recursiva se obtiene que

\begin{minipage}{0.45\linewidth}
    para $sin(x)$ es
    \begin{align*}
        P_0 & = a_n                  \\
        P_1 & = a_{n-1} + x^2P_0     \\
        P_2 & = a_{n-2} + x^2P_1     \\
            & \qquad\vdots           \\
        P_i & = a_{n-i} + x^2P_{i-1} \\
            & \qquad\vdots           \\
        P_n & = xP_n
    \end{align*}
\end{minipage}
\begin{minipage}{0.45\linewidth}
    En el caso de $cos(x)$ es:
    \begin{align*}
        P_0 & = a_n                  \\
        P_1 & = a_{n-1} + x^2P_0     \\
        P_2 & = a_{n-2} + x^2P_1     \\
            & \qquad\vdots           \\
        P_i & = a_{n-i} + x^2P_{i-1} \\
            & \qquad\vdots           \\
        P_n & = a_{n-1}+x^2P_{n-1}
    \end{align*}
\end{minipage}

Por lo tanto, la ecuación que se calculara es la siguiente:
\begin{equation}
    f_c(x) = \frac{x\left(\sum\limits_{i=0}^{n=2} \frac{(-1)^i}{(2i)!} x^{2i}\right)- \left(\sum\limits_{i=0}^{n=2} \frac{(-1)^i}{(2i+1)!}x^{2i+1}\right) }{x-\sum\limits_{i=0}^{n=2} \frac{(-1)^i}{(2i+1)!} x^{2i+1}}
    \label{eq:problema_fc}
\end{equation}

Cada término de la ecuación \ref{eq:problema_fc} fue redondeado como en la ecuación \ref{eq:problema2b}. A su vez cada término de las series fue redondeado a cuatro decimales usando la función \textcolor{citecolor}{round\_custom()}.
Evaluando la función en $x=0.1$, se obtiene como resultado que $f(x=0.1)=-1.,5000$.