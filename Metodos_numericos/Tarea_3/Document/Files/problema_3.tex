\section*{Problema 3}
\textbf{Find the number of terms of the exponential series such that their sum gives the value of e\textsuperscript{x} correct to six decimal places at x = 1.}

El procedimiento para encontrar la respuesta a este problema fue el siguiente:
\begin{lstlisting}[language=python]
    n_term = 1
    fx = round(exp(1.0),6)
    fx_approx = round(approx(x,n_term),6)
    while fx != fx_approx:
        n_term++
        fx_approx = round(approx(x,n_term),6)
\end{lstlisting}

El número de terminos de la serie de la función $f(x)=e^x$ necesaria para que coincida su valor numérico a seis decimales es 9.

El programa de este problema se encuentra en la carpeta \textcolor{citecolor}{Problema\_3}. El comando para compilarlo es el siguiente:

\begin{lstlisting}[language=bash]
    gcc -Wall -o main.out main.c -lm -std=c11
\end{lstlisting}

El output esperado del programa es el siguiente:
\begin{lstlisting}[language=bash]
    -----------------------------------
    n	f(x)		f_approx(x)
    1	2.718282	2.000000
    2	2.718282	2.500000
    3	2.718282	2.666667
    4	2.718282	2.708334
    5	2.718282	2.716667
    6	2.718282	2.718056
    7	2.718282	2.718254
    8	2.718282	2.718279
    9	2.718282	2.718282

    Se obtuvo la igualdad usando 9 terminos de la serie
\end{lstlisting}