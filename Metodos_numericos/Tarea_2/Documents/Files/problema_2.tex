\item \textbf{Implement a function to comput
    \begin{equation*}
        f(x)=\frac{1}{\sqrt{x^2+1}-x}
    \end{equation*}
    When evaluating the previous function we can lose accuracy, transform the right hand side to avoid error (or improve the accuracy). Implement the transformed expression and compare the results with the original function. Note: use values of x greater than 10000.}\\
Se crearon una serie de funciones donde se calculará $f(x)$, como entrada recibiran un número del tipo double y daran de salida un número float, double y long double. Los resultados de cada función con diferentes valores de x se encuentran enlistados en la tabla \ref{table:problema2}.
\begin{table}[H]
    \centering
    \begin{tabular}{llll} \hline
        \textbf{x}         & \textbf{float f(x)} & \textbf{double f(x)} & \textbf{long double f(x)} \\ \hline
        1.000000           & 2.414214            & 2.414214             & 2.414214                  \\
        10.000000          & 20.050020           & 20.049876            & 20.049876                 \\
        100.000000         & 200.109924          & 200.005000           & 200.005000                \\
        1000.000000        & 2048.000000         & 2000.000500          & 2000.000500               \\
        10000.000000       & inf                 & 19999.999778         & 20000.000050              \\
        100000.000000      & inf                 & 200000.223331        & 199999.999937             \\
        1000000.000000     & inf                 & 1999984.771129       & 2000000.005050            \\
        10000000.000000    & inf                 & 19884107.851852      & 19999847.711292           \\
        100000000.000000   & inf                 & inf                  & 200056700.832606          \\
        10000000000.000000 & inf                 & inf                  & inf                       \\ \hline
    \end{tabular}
    \caption{Valores de x para las diferences funciones.}
    \label{table:problema2}
\end{table}