\section*{Problema 1}
\textbf{Find the fourth Taylor polynomial $P_4(x)$ for the function $f (x) = xe^{x^2}$ about $x_0 = 0$.}

Como se sabe que
\begin{equation*}
    e^x= \sum_{i=0}^n \frac{x^i}{i!}
\end{equation*}
entonces
\begin{align*}
    f(x) & =xe^{x^2}                                       \\
         & =x \left(\sum_{i=0}^n \frac{(x^2)^i}{i!}\right) \\
         & =  \sum_{i=0}^n x\left(\frac{x^{2i}}{i!}\right) \\
         & = \sum_{i=0}^n \frac{x^{2i+1}}{i!}
\end{align*}
por lo tanto, la función a implementar es:
\begin{equation*}
    f(x)= \sum_{i=0}^n \frac{x^{2i+1}}{i!}
\end{equation*}
Expandiendo esta suma hasta $n=3$, se obtiene que:
\begin{align*}
    f(x) & = a_0 x + a_1 x^3 + a_2 x^5 + a_3x^7 \\
         & = x(a_0 + x^2(a_1+x^2(a_2+a_3x^2)))
\end{align*}
por lo que la función recursiva es:
\begin{align*}
    P_0 & = a_3            \\
    P_1 & = a_2+x^2P_0     \\
    P_2 & = a_1+x^2P_1     \\
    P_3 & = x(a_0 +x^2P_2)
\end{align*}
\begin{enumerate}
    \item \textbf{Find an upper bound for $|f (x)-P_4 (x)|$, for $0 \leq x \leq 0.4$, ie find an upper bound of $|R_4 (x)|$ for $0 \leq x \leq 0.4$}

          Los límites superiores que se obtuvieron se encuentran en la tabla \ref{table:problema1}.
          \begin{table}[H]
              \centering
              \begin{tabular}{lc} \hline
                  \textbf{Operación} & \textbf{Límite superior} \\ \hline
                  $|f(x)-P_4(x)|$    & 0.0000003595             \\
                  $|R_4(x)|$         & 0.0000003499             \\ \hline
              \end{tabular}
              \caption{Límites superiores para $|f(x)-P_4(x)|$ y $|R_4(x)|$.}
              \label{table:problema1}
          \end{table}
    \item \textbf{Approximate $\int_0^{0.4} f(x)dx$ using $\int_0^{0.4} P_4(x)dx$}

          El resultado de la integral usando el polinomio $P_4(x)$ es de 0.086784.
\end{enumerate}
Con esto se comprueba la precisión que se puede obtener al escribir una función como serie de potencias.
El programa de este problema se encuentra en la carpeta \textcolor{citecolor}{Problema\_1}. El comando para compilarlo es el siguiente:

\begin{lstlisting}[language=bash]
    gcc -Wall -o main.out main.c -lm -std=c11
\end{lstlisting}
