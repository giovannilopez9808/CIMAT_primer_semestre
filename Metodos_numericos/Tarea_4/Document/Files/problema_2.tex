\section*{Problema 2}
\textbf{Use the previous methods to compute a zero of the following functions:}
\begin{align}
    e^x+2^{-x} + 2cos(x) - 6 & =0  \qquad 1\leq   x \leq 2  \label{eq:function1_problema2} \\
    ln(x1)+cos(x-1)          & =0  \qquad 1.3\leq x \leq 2 \label{eq:function2_problema2}
\end{align}

El parámetro de entrada para el método de Newton es el valor intermedio del intervalo definido como:

\begin{equation*}
    x_{m}= \frac{x_1-x_0}{2}
\end{equation*}

Usando los métodos del problema 1 se obtuvieron los resultados mostrados en la tabla \ref{table:solutions_problema2}.

\begin{table}[H]
    \centering
    \begin{tabular}{cccc} \hline
        \textbf{Función}                     & \textbf{Bisección} & \textbf{Newton} & \textbf{Secante} \\ \hline
        Función \ref{eq:function1_problema2} & 1.829383           & 1.829384        & 1.829384         \\
        Función \ref{eq:function2_problema2} & 1.397748           & 1.397748        & 1.397748         \\ \hline
    \end{tabular}
    \caption{Soluciones a las ecuaciones \ref{eq:function1_problema2} y \ref{eq:function2_problema2} usando los métodos de bisección, Newton y secante.}
    \label{table:solutions_problema2}
\end{table}


El output esperado del programa es el siguiente:
\begin{lstlisting}[language=bash]
    ----------------------------------------
    Raices de la funcion 1 en el intervalo
    [1.000000,2.000000] con el metodo:
        Biseccion:
        x = 1.829383
        Newton:
        x = 1.829384
        Secante:
        x = 1.829384
    
    ----------------------------------------
    Raices de la funcion 2 en el intervalo
    [1.300000,2.000000] con el metodo:
        Biseccion:
            x = 1.397748
            Newton:
            x = 1.397748
        Secante:
            x = 1.397748
\end{lstlisting}
El programa se encuentra en la carpeta \textcolor{citecolor}{Problema\_2}. Para compilar el programa se debe ingresar la siguiente linea:

\begin{lstlisting}[language=bash]
    gcc -Wall -Wextra -Werror -pedantic -ansi -o main.out main.c -lm -std=c11  
\end{lstlisting}
