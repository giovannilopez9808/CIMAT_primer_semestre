\subsection*{Inciso c}

El algoritmo de newton moficado haciendo uso de la ecuación \ref{eq:h_problema1} se encuentra en la carpeta \folder{Problema\_1}.

Se uso como ejemplo la ecuación \ref{eq:f_problem1_incisoc}. En la tabla  \ref{table:problem1_parameters} se encuentran los parámetros usados para este ejemplo.


\begin{equation}
    f(x) = e^{2x-2}-2x+1 \label{eq:f_problem1_incisoc}
\end{equation}

\begin{table}[H]
    \centering
    \begin{tabular}{ccc} \hline
        $\mathbf{x_0}$ & \textbf{tol\_x} & \textbf{tol\_f} \\   \hline
        -10            & $10^{-6}$       & $10^{-6}$       \\  \hline
    \end{tabular}
    \caption{Parámetros usados para el ejemplo de la ecuación \ref{eq:f_problem1_incisoc}}
    \label{table:problem1_parameters}
\end{table}

Los resultados para cada iteración se encuentran en la tabla \ref{table:problema1_results}.
\begin{table}[H]
    \centering
    \begin{tabular}{lcc} \hline
        \textbf{Iteración} & $\mathbf{x}$ & $\mathbf{|f(x)|}$ \\ \hline
        1                  & 0.500000     & 0.367879          \\
        2                  & 0.940018     & 0.006916          \\
        3                  & 0.998848     & 0.000003          \\
        4                  & 1.000000     & 0.000000          \\ \hline
    \end{tabular}
    \caption{Resultados del programa usando la ecuación \ref{eq:f_problem1_incisoc}}
    \label{table:problema1_results}
\end{table}
