\subsection*{Inciso a}

\textbf{Muestra que si $f\in C^1 [a,b]$ tiene un cero simple en $x'\in(a,b)$, entonces $f'(x')\neq 0$.}

Se tiene que que $f(x)$ se puede escribir como la ecuacion \ref{eq:f_problema1}. Como $x'$ es un cero simple, entonces $m=1$, por lo tanto $f$ es:

\begin{equation*}
    f(x)=(x-x')q(x)
\end{equation*}

Calculando $f'$, se tiene que:

\begin{equation*}
    f'(x)=q(x)-(x-x')q'(x)
\end{equation*}

evaluando $f'(x')$, se obtiene que:

\begin{equation*}
    f'(x') = q(x')
\end{equation*}

como $q(x')\neq 0$, por lo tanto $f'(x')\neq 0$.

