\subsection*{Inciso b}

\textbf{Se cumple que el recíproco del inciso a, es decir, si $f\in C^1[a,b], x' \in (a,b), f(x')=0$ y $f'(x')\neq 0$ entonces $x'$ es un cero simple de $f$. Basado en esta afirmación, muestra que si la función $f(x)$ tiene un cero $x'$ de multiplicidad $m>1$, entonces $x'$ es un cero simple de la función}

\begin{equation}
    h(x) = \frac{f(x)}{f'(x)} \label{eq:h_problema1}
\end{equation}

Se tiene que $f$ puede escribirse como la ecuación \ref{eq:f_problema1}. Entonces, calculando su derivada se obtiene que:

\begin{align*}
    f'(x) & = m(x-x')^{m-1} q(x) +(x-x')^mq(x)                                    \\
          & = (x-x')^{m}\left [  \left (\frac{m}{x-x'}\right ) q(x)+q'(x)\right ] \\
          & = (x-x')^m \left ( \frac{mq(x)+q'(x)(x-x')}{x-x'}\right )
\end{align*}

Entonces, escribiendo la ecuación \ref{eq:h_problema1}. Se obtiene que:

\begin{align*}
    h(x) & = \frac{f(x)}{f'(x)}                                                           \\
         & = \frac{(x-x')^mq(x)}{(x-x')^m \left ( \frac{mq(x)+q'(x)(x-x')}{x-x'}\right )} \\
         & = \frac{q(x)(x-x')}{mq(x)+(x-x')q'(x)}
\end{align*}

Encontrando los valores para los cuales $h(x)=0$, se obtiene que:

\begin{align*}
    \frac{q(x)(x-x')}{mq(x)+(x-x')q'(x)} & = 0 \\
    q(x)(x-x')                           & = 0 \\
    q(x) =0 \qquad x-x'                  & =0  \\
\end{align*}

En donde se obtiene que un cero de $h$ es $x'$ con multiplicidad 1.