\subsection{Ecuaciones diferenciales parciales}

Una ecuación diferencial parcial (PDE, por sus siglas en ingles) es una ecuación que relaciona la dependencia de una función $u$ con sus derivadas parciales\cite{sommerfeld_1949}. Las PDE pueden ser clasificadas en homogéneas y no homogéneas. Una PDE es homogénea si cada término contiene alguna dependencia de la variable $u$ o sus derivadas parciales.

\subsection{Condiciones de contorno}

El uso de una solución particular de una PDE es más frecuente que el uso de la solución general. Dada una PDE y sus valores en un dominio acotado $D$\cite{wazwaz_2002}. Una condición de contorno es denominada homogénea cuandos sus valores son iguales a cero, de cualquier otra manera, la condicion de contorno es denominada no homogénea. La función $u$ usualmente esta definida en el dominio $D$. La información del contorno $D$ es llamada condiciones de contorno. Las condiciones de contorno estan definidas en tres tipos.

\begin{itemize}
    \item \textbf{Condiciones de contorno de Dirichlet}

          En este tipo de condiciones de contorno, la función $u$ es normalmente definida en el contorno $D$.
    \item \textbf{Condiciones de contorno de Neumann}

          El tipo de condiciones de contorno de Neumann son aquellas donde las derivadas de $u$ estan definidas en el contorno $D$.
    \item \textbf{Condiciones de contorno mixtas}

          Las condiciones de contorno mixtas son aquellas que utilizan las condiciones de contorno de Dirichlet y Neumann. Con esto, se tiene a la función $u$ y su derivada definida en el contorno $D$.
\end{itemize}

\subsubsection{Condiciones iniciales}

La mayoria de las PDE son aplicadas para describir la dinámica de un sistema físico. Este conjunto de PDE dependen de una variable temporal $t$. Por lo que, cuando la función $u$ se encuentra descrita en $t=0$, se dice que $u$ tiene una condicióm inicial.