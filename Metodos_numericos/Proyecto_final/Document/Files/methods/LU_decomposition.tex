\subsection{Factorización LU de una matriz triagonal\label{sec:LU_factorization}}

Sea una matriz de tamaño $n \times n$, tal que su forma esta descrita como en la ecuación \ref{eq:tri_matrix}\cite{el_2003}.

\begin{equation}
    A= \begin{pmatrix}
        b_1    & c_1    & 0      & 0       & 0       & 0       \\
        a_2    & b_2    & c_2    & 0       & 0       & 0       \\
        0      & a_3    & b_3    & c_3     & 0       & 0       \\
        0      & 0      & \ddots & \ddots  & \ddots  & 0       \\
        \vdots & \vdots & \ddots & a_{n-1} & b_{n-1} & c_{n-1} \\
        0      & 0      & \cdots & 0       & a_n     & b_n
    \end{pmatrix}
    \label{eq:tri_matrix}
\end{equation}

entonces, esta se puede descomponer en dos matrices diagonales, de tal forma que:
\scriptsize
\begin{equation}
    A = \begin{bmatrix}
        b_1    & c_1    & 0      & 0       & 0       & 0       \\
        a_2    & b_2    & c_2    & 0       & 0       & 0       \\
        0      & a_3    & b_3    & c_3     & 0       & 0       \\
        0      & 0      & \ddots & \ddots  & \ddots  & 0       \\
        \vdots & \vdots & \ddots & a_{n-1} & b_{n-1} & c_{n-1} \\
        0      & 0      & \cdots & 0       & a_n     & b_n
    \end{bmatrix} =
    \begin{bmatrix}
        1        & 0        & 0      & 0            & 0        & 0 \\
        \alpha_2 & 1        & 0      & 0            & 0            \\
        0        & \alpha_3 & 1      & 0            & 0        & 0 \\
        0        & 0        & \ddots & \ddots       & \ddots   & 0 \\
        \vdots   & \vdots   & \ddots & \alpha_{n-1} & 1        & 0 \\
        0        & 0        & \cdots & 0            & \alpha_n & 1
    \end{bmatrix}\begin{bmatrix}
        \beta_1 & c_1     & 0       & 0      & 0           & 0       \\
        0       & \beta_2 & c_2     & 0      & 0           & 0       \\
        0       & 0       & \beta_3 & c_3    & 0           & 0       \\
        0       & 0       & \ddots  & \ddots & \ddots      & 0       \\
        \vdots  & \vdots  & \ddots  & 0      & \beta_{n-1} & c_{n-1} \\
        0       & 0       & \cdots  & 0      & 0           & \beta_n
    \end{bmatrix}
\end{equation}
\normalsize

donde $b_1 = \beta_1$, por ende, los términos $a_j$ y $b_j$ pueden ser calculados con las ecuaciones \ref{eq:a_j} y \ref{eq:b_j}.

\begin{align}
    a_j = \alpha_j \beta_{j-1}  \label{eq:a_j} \\
    b_j = \alpha_j c_{j-1} + \beta_j \label{eq:b_j}
\end{align}