\section{Introducción}

Una ecuación diferencial parcial relaciona a una función con más de una variable a partir de derivadas parciales. La ecuación de transferencia de calor es una ecuación diferencial parcial que desceibe la variación de la temperatura dada una región en un periodo de tiempo. El método de variables separables es un método análitico para resolver de manera práctica una ecuación diferencial ordinaria o parcial. A pesar de la téoria que se tiene para la solución análitica de una ecuación diferencial, esta no es flexible a la hora de modificar datos de la ecuación diferencial. Es por ello que la aplicación de métodos numéricos se convierte en una gran herramienta para obtener una aproximación de las soluciones\cite{chavarria_2019}. Existen dos métodos numéricos para obtener la aproximación de la solución a una ecuación diferencial. El método de elementos finitos y el método de diferencias finitas\cite{quintana_2016}.