\fbckg{background/white}
\begin{frame}
    \Huge
    \misc{Ecuaciones diferenciales\\parciales}
\end{frame}

\begin{frame}
    \misc{
        \textbf{Ecuación diferencial parcial:}\\
        Una ecuación que contiene derivadas parciales de una o más variables dependientes de dos o más variables independientes.
        \begin{align*}
            \frac{\partial u }{\partial t} = \alpha^2 \frac{\partial^2 u}{\partial x^2} & \qquad \frac{\partial^2 u}{\partial x^2} + \frac{\partial^2 u}{\partial y^2} = f(x,y) \\
        \end{align*}
    }
\end{frame}

\begin{frame}
    \misc{
        \textbf{Condiciones de contorno}\\
        En el caso que se tenga una función dependiente de dos variables como $u(x,t)$.
        \begin{itemize}
            \item \textbf{Dirichlet}
                  \begin{equation*}
                      u(a,t)=g(a) \qquad u(b,t)=h(b) \qquad a<x<b
                  \end{equation*}

            \item \textbf{Neumann}
                  \begin{equation*}
                      \frac{d u(a,t)}{dx}=g(a) \qquad \frac{d u(b,t)}{dx}=h(b) \qquad a<x<b
                  \end{equation*}

            \item \textbf{Mixtas}
                  El sistema se encuentra bajo las condiciones de contorno de Neumann y Dirichlet al mismo tiempo.
        \end{itemize}
    }
\end{frame}