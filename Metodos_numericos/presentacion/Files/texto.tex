\begin{table}[H]
	\centering
	\begin{tabular}{lll} \hline
		Día       & Horario       & Salón \\ \hline
		Lunes     & 11:00 a 12:20 & G101  \\
		Miercoles & 11:00 a 12:20 & G101  \\
		Viernes   & 9:30 a 10:50  & K201  \\ \hline
	\end{tabular}
	\caption{Horario de clases de la materia}
\end{table}

\section*{Temario}
\begin{enumerate}
	\item \textbf{Introducción}
	      \begin{enumerate}
		      \item Preliminares
		      \item Problemas no lineales en una variable
		            \begin{itemize}
			            \item Solución de ecuaciones. Bisección, método de Newton
			            \item Minimización de funciones
		            \end{itemize}
	      \end{enumerate}
	\item \textbf{Álgebra lineal numérica}
	      \begin{enumerate}
		      \item Solución de sistemas lineales
		            \begin{itemize}
			            \item Eliminación Gaussiana, sustitución hacia atrás
			            \item Descomposición LU, QR, inversa y determinante de una matriz
			            \item Métodos iterativos, Jacobi, Gauss-Seidel, gradiente conjugado
			            \item Precondicionadores de solvers iterativos
		            \end{itemize}
		      \item El problema de valores propios
		            \begin{itemize}
			            \item Método de Jacobi
			            \item Método de la potencia
			            \item El problema generalizado de valores propios
		            \end{itemize}
	      \end{enumerate}
	\item \textbf{Métodos numéricos en cálculo}
	      \begin{enumerate}
		      \item Interpolación
		            \begin{itemize}
			            \item Polinomial
			            \item Splines cúbicos
		            \end{itemize}
		      \item Integración y diferenciación
		            \begin{itemize}
			            \item Diferencias finitas
			            \item Métodos clásicos integración
			            \item Método de Romberg
		            \end{itemize}
		      \item Problemas no lineales y polinomios ortogonales
		            \begin{itemize}
			            \item Sistemas no lineales. Métodos cuasi-Newton
			            \item Minimización de funciones
		            \end{itemize}
	      \end{enumerate}
	\item \textbf{Ecuaciones diferenciales}
	      \begin{enumerate}
		      \item Problemas con valores iniciales
		            \begin{itemize}
			            \item Método de Euler
			            \item Métodos Runge-Kutta, otros métodos
		            \end{itemize}
		      \item Problemas con valores a la fontera
		            \begin{itemize}
			            \item Diferencias finitas
			            \item Elemento finito
			            \item Problemas de advección-difusión
			            \item Problemas de valores propios
		            \end{itemize}
	      \end{enumerate}
\end{enumerate}

\section*{Evaluaciones}

\begin{table}[H]
	\centering
	\begin{tabular}{ll}\hline
		Actividad    & Porcentaje \\ \hline
		Tareas       & 50\%       \\
		Examenes (2) & 50\%       \\ \hline
	\end{tabular}
	\caption{Evaluaciones en el semestre}
\end{table}

\section*{Lineamientos generales}
\subsection*{Código}
\begin{itemize}
	\item El lenguaje de programación es C/C++ para estudiantes de computación
	\item Se debe enviar código, librerías y archivos de entrada
	\item Código incluirá comentarios
	\item Deberá de contar con un ejemplo para ejecutarse
	\item Deberá ser interactivo pasando parámetros
	\item Incluir ayuda para la ejecucción
\end{itemize}
