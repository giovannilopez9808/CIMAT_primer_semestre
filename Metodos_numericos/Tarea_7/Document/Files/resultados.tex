\section{Resultados}

\subsection{Factorización LU - Crout}

\subsubsection{Matriz de prueba}

Se realizaron diferentes pruebas para el algoritmo de factorización LU usando la versión de Crout. La primera fue realizada con la matriz de la ecuación \ref{eq:matrix_test_LU}.

\begin{equation}
    A= \begin{pmatrix}
        3  & -1 & 4  & -1 \\
        -1 & -1 & 3  & 1  \\
        2  & 3  & -1 & -1 \\
        7  & 1  & 1  & 2
    \end{pmatrix} \qquad
    B = \begin{pmatrix}
        10 \\
        5  \\
        1  \\
        -20
    \end{pmatrix}
    \label{eq:matrix_test_LU}
\end{equation}

Se obtuvo la factorización es la siguiente:

\begin{equation}
    A = \begin{pmatrix}
        3  & 0     & 0    & 0    \\
        -1 & -1.34 & 0    & 0    \\
        2  & 3.67  & 8.25 & 0    \\
        7  & 3.34  & 2.5  & 5.54
    \end{pmatrix}
    \begin{pmatrix}
        1 & -0.34 & 1.34  & -0.34 \\
        0 & 1     & -3.25 & -0.5  \\
        0 & 0     & 1     & 0.18  \\
        0 & 0     & 0     & 1
    \end{pmatrix}
\end{equation}


Y la solución del sistema de ecuaciones asociado es el siguiente:

\begin{equation}
    \begin{cases}
        x_1 & = -2 \\
        x_2 & = 1  \\
        x_3 & = 3  \\
        x_4 & = -5
    \end{cases}
    \label{eq:LU_test}
\end{equation}

El sistema de la ecuación \ref{eq:matrix_test_LU} fue resuelto en anteriores tareas y se obtuvieron los mismo resultados expuestos en las ecuaciones \ref{eq:LU_test}.

\subsubsection{Archivos LARGE.txt}

Usando los archivos \file{M\_LARGE.txt} y \file{V\_LARGE.txt} se obtuvieron los resultados contenidos en el archivo \file{Solution\_Large.txt}, el cual se encuentra en la carpeta \folder{Problema\_1}.

\subsubsection{Archivos SMALL.txt \label{sec:small_lu}}

El sistema de ecuaciones mostrado en la ecuación \ref{eq:v_small_matrix_LU} esta contenida en los archivos \file{M\_SMALL.txt} y \file{V\_SMALL.txt}.

\begin{equation}
    A= \begin{pmatrix}
        2.402822 & 4.425232 & 1.929374 & 1.370355 \\
        1.201411 & 2.212616 & 0.964687 & 0.685178 \\
        1.119958 & 0.964687 & 2.053172 & 0.566574 \\
        0.742142 & 0.685178 & 0.566574 & 1.696828
    \end{pmatrix} \qquad
    B = \begin{pmatrix}
        0.060000 \\
        0.542716 \\
        0.857204 \\
        0.761270
    \end{pmatrix}
    \label{eq:v_small_matrix_LU}
\end{equation}

Al introducir la matriz a el proceso de factorización LU con la versión de Crout da error, ya que un termino de la diagonal de la matriz L es igual a cero, por ende provocando una indeterminación. Por lo tanto, no es posible resolver el sistema de ecuaciones \ref{eq:v_small_matrix_LU} por medio de la factorización LU. El programa con la información de la matriz se encuentra en la carpeta \folder{Problema\_2}


\subsection{Factorización Cholesky}

\subsubsection{Matriz de prueba}

Se probo el metodo de factorización de Cholesky con la matriz de la ecuación \ref{eq:test_cholesky}.

\begin{equation}
    A= \begin{pmatrix}
        1  & -1 & 1  \\
        -1 & 5  & -5 \\
        1  & -5 & 6
    \end{pmatrix}
    \label{eq:test_cholesky}
\end{equation}

El programa que ejecuta este proceso es el contenido en la carpeta \folder{Problema\_3\_test}, el resultado que se obtuvo es la ecuación \ref{eq:test_cholesky_output}, el cual se encuentra en los archivos \file{L\_test.txt} y \file{LT\_test.txt}.

\begin{equation}
    A = \begin{pmatrix}
        1  & 0  & 0 \\
        -1 & 2  & 0 \\
        1  & -2 & 1
    \end{pmatrix}
    \begin{pmatrix}
        1 & -1 & 1  \\
        0 & 2  & -2 \\
        0 & 0  & 1
    \end{pmatrix}
    \label{eq:test_cholesky_output}
\end{equation}

\subsubsection{Matrices dadas \label{sec:cholesky}}

Se factorizo las matrices de tamaño n$\times$n, donde $n\in \{ 4,50,100\}$, donde los elementos de la matriz son los descritos en la ecuación

\begin{equation}
    a_{ij} = \begin{cases}
        2,  & \text{si } i=j     \\
        -1, & \text{si } |i-j|=1 \\
        0,  & \text{otro}
    \end{cases}
\end{equation}

La cual es generada en la función \script{create\_matrix} que se encuentra en el archivo \file{functions.h} en la carpeta \folder{Problema\_3\_1}. Los resultados para los diferentes valores de n se encuentran en los archivos \file{L\_n.txt} y \file{LT\_n.txt} en la carpeta \folder{Problema\_3\_1}.