\subsection{Factorización LU - Crout}

Si A es una matriz de tamaño n$\times$n, entonces su factorización en matrices L y U deben seguir la forma de las ecuaciones \ref{eq:matrix_L_Crout} y \ref{eq:matrix_U_Crout}.

\begin{equation}
    L = \begin{pmatrix}
        l_{11} & 0      & \cdots & 0      \\
        l_{21} & l_{22} & \cdots & 0      \\
        \vdots & \vdots & \ddots & 0      \\
        l_{n1} & l_{n2} & \cdots & l_{nn}
    \end{pmatrix}
    \label{eq:matrix_L_Crout}
\end{equation}

\begin{equation}
    U = \begin{pmatrix}
        1      & u_{12} & \cdots & u_{1n} \\
        0      & 1      & \cdots & u_{2n} \\
        \vdots & \vdots & \ddots & \vdots \\
        0      & 0      & \cdots & 1
    \end{pmatrix}
    \label{eq:matrix_U_Crout}
\end{equation}

Para calcular los elementos de cada matriz se tiene el conjunto de ecuaciones \ref{eq:equations_crout}.

\begin{equation}
    \begin{cases}
        l_{ij} = a_{ij} - \sum\limits_{k=1}^{j-1} l_{ik}u_{kj} & \\
        u_{ij} = \frac{a_{ij}-\sum\limits_{k=1}^{i-1}l_{ik}u_{kj}}{l_{ii}}
    \end{cases}
    \label{eq:equations_crout}
\end{equation}

El algoritmo planteado para la solución de este problema se encuentran en las carpetas \folder{Problema\_1} y \folder{Problema\_2}, en el archivo \file{LU\_decomposition.h} en la función \script{obtain\_LU\_crout}.

Al tener los elementos determiandos de las matrices L y U, es más sencillo realizar la solución a un sistema de ecuacuiones, ya que, Al tratarse de matrices triangulares sus algoritmos son más faciles de implementar. Suponiendo que se tiene el sistema $Ax=B$, entonces:

\begin{align*}
    Ax  & =B \\
    LUx & =B \\
\end{align*}

nombrando
\begin{equation}
    Ux=y \label{eq:uxy}
\end{equation}
, entonces:

\begin{equation}
    Ly=B
    \label{eq:lyb}
\end{equation}

donde las ecuaciones \ref{eq:uxy} y \ref{eq:lyb}  son dos sistemas de ecuaciones lineales del tipo triangular.

El algoritmo planteado para realizar la factorización LU con la version de Crount es la siguiente:

\begin{lstlisting}[style=CStyle]
    // input: matriz
    // output: L,U
    for (int i = 0; i < n; i++)
    {
        for (int j = i; j < n; j++)
        {
            sum_ij = 0;
            for (int k = 0; k <= i - 1; k++)
            {
                sum_ij += l_ik * u_kj;
            }
            l_ij = matrix_ij - sum_ij;
        }
        for (int j = i + 1; j < n; j++)
        {
            sum_ij = 0;
            for (int k = 0; k < j - 1; k++)
            {
                sum_ij += l_ik * u_kj;
            }
            u_ij = (matrix_ij - sum_ij) / l_ii;
        }
    }
\end{lstlisting}

Al termino de este de implementaron las funciones \script{solve\_triangular\_inferior\_matrix} y \script{solve\_tri\-angular\_superior\_matrix}.