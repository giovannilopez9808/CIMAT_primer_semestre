\subsection{Factorización Cholesky}

Sea A, una matriz simetica y definida positiva puede ser factorizada en la forma LU con $U=L^T$, esto es descrito en la ecuacuión \ref{eq:cholesky_def}.

\begin{equation}
    A=LL^T \label{eq:cholesky_def}
\end{equation}

El algoritmo emplea el conjunto de ecuaciones .

\begin{equation}
    \begin{cases}
        l_{ii} = \sqrt{a_{ii}- \sum\limits_{k=1}^{i-1} l_{ik}^2}           & \\
        l_{ij} = \frac{a_{ij}-\sum\limits_{k=1}^{j-1}l_{ik}l_{kj}}{l_{jj}} &
    \end{cases}
\end{equation}

El algoritmo de factorización de Cholesky se encuntra en la carpeta \folder{Problema\_3} en el archivo \file{Cholesky\_decomposition.h} en la función \script{obtain\_Cholesky}. La función es la siguiente:

\begin{lstlisting}[style=CStyle]
    // inputs: matrix, n
    // output
    for (int i = 0; i < n; i++)
    {
        sum = 0;
        for (int j = 0; j < i; j++)
        {
            sum += l_ik * l_ik;
        }
        sqrt_number = matrix_ii - sum;
        vadidate_positive_matrix(sqrt_number);
        validate_l_ii(l_ii);
        for (int j = i; j < n; j++)
        {
            sum = 0;
            for (int k = 0; k < i; k++)
            {
                sum += l_ik * l_jk;
            }
            l_ij = (matrix_ij - sum) / l_ii;
        }
    }
    fill_L_transpose(l, lt);
\end{lstlisting}