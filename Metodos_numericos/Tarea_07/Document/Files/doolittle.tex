\subsection{Factorización Doolittle}

El método de factorización se basa en la eliminación Gaussiana. A partir de una matriz de tamaño n$\times$n se obtiene una matriz triangular superior realizando operaciones entre columnas y renglones de la misma matriz. Entonces, la factorización de Doolittle se concentra en encontrar la matriz triangular inferior tal que se cumple la ecuación \ref{eq:definition_doolittle}.

\begin{equation}
    \begin{pmatrix}
        1      & 0      & \cdots & 0      \\
        l_{21} & 1      & \cdots & 0      \\
        l_{31} & l_{32} & \cdots & 0      \\
        \vdots & \vdots & \ddots & \vdots \\
        l_{n1} & l_{n2} & \cdot  & 1
    \end{pmatrix}
    \begin{pmatrix}
        u_{11} & u_{12} & \cdots & u_{1n} \\
        0      & u_{22} & \cdots & u_{2n} \\
        0      & 0      & \cdots & u_{3n} \\
        \vdots & \vdots & \ddots & \vdots \\
        0      & 0      & \cdot  & u_{nn}
    \end{pmatrix}=
    \begin{pmatrix}
        a_{11} & a_{12} & \cdots & a_{1n} \\
        a_{21} & a_{22} & \cdots & a_{2n} \\
        a_{31} & a_{32} & \cdots & a_{3n} \\
        \vdots & \vdots & \ddots & \vdots \\
        a_{n1} & a_{n2} & \cdot  & a_{nn}
    \end{pmatrix}
    \label{eq:definition_doolittle}
\end{equation}

Realizando la multiplicación de matrices se obtiene que:

\begin{equation}
    \begin{pmatrix}
        u_{11} & u_{12}                     & \cdots & u_{1n}                                  \\
        l_{21} & l_{21}u_{12}+u_{22}        & \cdots & l_{21}u_{1n}+u_{2n}                     \\
        l_{31} & l_{31}u_{12}+l_{32}u_{22}  & \cdots & l_{31}u_{1n}+l_{3n}u_{22}+u_{3n}        \\
        \vdots & \vdots                     & \ddots & \vdots                                  \\
        l_{n1} & l_{n1}u_{12}+l_{n2}+u_{22} & \cdots & l_{n1}u_{1n}+l_{n2}u_{2n}+\cdots+u_{nn}
    \end{pmatrix}=
    \begin{pmatrix}
        a_{11} & a_{12} & \cdots & a_{1n} \\
        a_{21} & a_{22} & \cdots & a_{2n} \\
        a_{31} & a_{32} & \cdots & a_{3n} \\
        \vdots & \vdots & \ddots & \vdots \\
        a_{n1} & a_{n2} & \cdot  & a_{nn}
    \end{pmatrix}
    \label{eq:multiplication_matrix}
\end{equation}

Observando la diagonal de la ecuación \ref{eq:multiplication_matrix}, se puede obtener la ecuación \ref{eq:diagonal}.

\begin{equation}
    u_{ii} = a_{ii} - \sum_{j=1}^i l_{ij}u_{ji}
    \label{eq:diagonal}
\end{equation}

Los elementos en los queda se obtiene un elemento de U sin realizar una multiplicación con un termino de L se obtiene la ecuación \ref{eq:uij}.

\begin{equation}
    u_{ij} = a_{ij} - \sum_{k=1}^{i-1} l_{ik}u_{ki} \label{eq:uij}
\end{equation}

De los elementos donde un elemento de L esta multiplicando a un termino de la diagonal de U, se obtiene la ecuación \ref{eq:lij}.

\begin{equation}
    l_{ij} = \frac{a_{ij}-\sum\limits_{k=1}^{j-1} l_{ik}u_{kj}}{u_{jj}} \label{eq:lij}
\end{equation}