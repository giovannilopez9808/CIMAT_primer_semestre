\subsection{Método de iteraciones simultaneas}

El método de iteraciones simultaneas se uso para calcular 6 y 25 eigenpares de las matrices contenidas en los archivos \file{Eigen\_125x125.txt} y \file{Eigen\_500x500.txt} respectivamente. Los resultados de este método de encuentran en la carpeta \folder{Sub\_space\_method}. En la tabla \ref{table:subspace_results} se encuentra un resumen de los parámetros y resultados obtenidos con este método.

\begin{table}[H]
    \centering
    \begin{tabular}{lcccc}
        \hline
        Archivo            & Tolerancia                 & \begin{tabular}[c]{@{}c@{}}Dimensión\\ de la matriz\end{tabular} & \begin{tabular}[c]{@{}c@{}}Dimensión\\ del subespacio\end{tabular} & \begin{tabular}[c]{@{}c@{}}Número de\\ iteraciones\end{tabular} \\ \hline
        Eigen\_125x125.txt & \multirow{2}{*}{$10^{-4}$} & 125                       & 6                         & 3                         \\
        Eigen\_500x500.txt &                            & 500                       & 25                        & 5841                      \\ \hline
    \end{tabular}
    \caption{Parámetros y resultados usando el método de iteraciones simultaneas}
    \label{table:subspace_results}
\end{table}