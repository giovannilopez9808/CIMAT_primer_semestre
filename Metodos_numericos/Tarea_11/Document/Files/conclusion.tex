\section{Conclusiones}

\subsection{Método de la potencia y cociente de Rayleigh}
El número de iteraciones realizadas por el método en las matrices es igual al obtenido en el metodo de potencias original. Esto es debido a que el procedimiento entre los dos métodos es muy semejante, ya que se basan en la proyección de un vector sobre otro. Por lo tanto, si se quisiera optar por uno o por otro dependería de la optimización en el tiempo de ejecución.

\subsection{Método de iteraciones simultaneas}

El método de iteraciones simultaneas se puede comparar con el método de deflación, en este caso, con el uso del método de potencias. Compararando el número de iteraciones de los dos métodos se obtiene que el método de iteraciones simultaneas es más eficiente que el método de deflación. La razón de esta diferencia es debido a que en el método de iteraciones simultaneas el calculo de los $m$ eigenpares se realiza en cada iteración.