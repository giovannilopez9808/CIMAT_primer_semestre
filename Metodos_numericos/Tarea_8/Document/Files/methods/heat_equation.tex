\subsection{Ecuación de calor}

Aplicando el método de diferencias finitas a la ecuación \ref{eq:heat_equation} obtenemos la ecuación \ref{eq:heat_diferencias_finitas}.

\begin{equation}
    \frac{k(u_{i+1}-2u_i+u_{i-1})}{\Delta x^2} + Q = 0 \label{eq:heat_diferencias_finitas}
\end{equation}

Para el caso de cinco nodos se obtiene el sistema de ecuaciones señalado en la ecuación \ref{eq:heat_system}.

\begin{equation}
    \begin{cases}
        u_0 + -2u_1 +u_2 = \frac{-Q\Delta x^2}{K} \\
        u_1 + -2u_2 +u_3 = \frac{-Q\Delta x^2}{K} \\
        u_2 + -2u_3 +u_4 = \frac{-Q\Delta x^2}{K} \\
    \end{cases} \label{eq:heat_system}
\end{equation}

El sistema de ecuaciones \ref{eq:heat_system} puede ser descrito la ecuación matricial \ref{eq:heat_matrix}.

\begin{equation}
    \begin{pmatrix}
        2  & -1 & 0  \\
        -1 & 2  & -1 \\
        0  & -1 & 2
    \end{pmatrix}
    \begin{pmatrix}
        u_1 \\
        u_2 \\
        u_3
    \end{pmatrix}
    =\begin{pmatrix}
        \frac{Q\Delta x^2}{k}+u_0 \\
        \frac{Q\Delta x^2}{k}     \\
        \frac{Q\Delta x^2}{k} + u_4
    \end{pmatrix}
    \label{eq:heat_matrix}
\end{equation}

Donde $u_0$ y $u_4$ son las condiciones de frontera del problema. Si expandemos el problmea a $n$ nodos, obtendremos que para describir el sistema como una ecuación matricial debemos seguir los siguientes parámetros. Para obtener la matriz del sistema se tiene que:

\begin{equation}
    A =
    \begin{cases}
        2  & \text{para }          i=j     \\
        -1 & \text{para }          |i-j|=1 \\
        0  & \text{en otro lado}
    \end{cases}
    \qquad
    B = \begin{cases}
        \frac{Q\Delta x^2}{k}+u_0 & \text{para } i=1    \\
        \frac{Q\Delta x^2}{k}+u_n & \text{para } i=n-1  \\
        \frac{Q\Delta x^2}{k}     & \text{en otro lado}
    \end{cases}
    \label{eq:heat_system_finish}
\end{equation}

donde $i,j \in\{1,2,3,\dots,n-1\}$. El sistema de la ecuación \ref{eq:heat_system_finish} describe una matriz tridiagonal, por lo que se obtara por resolver el sistema usando la factorización por Cholesky.

El algoritmo que resuelve esta EDP es el siguiente:

\begin{lstlisting}[style=CStyle]
    // inputs: k, Q, L, u_0, u_n, n
    // output: solutions
    matrix,results = create_matrix_system(Q,k,l,n)
    L = Cholesky(matrix)
    solutions_y = solve_triangular_inferior(L,results)
    solutions_x = solve_triangular_superior(LT,solutions_y)
\end{lstlisting}

La funciones \script{Cholesky}, \script{solve\_triangular\_inferior} y \script{solve\_triangular\_superior} fueron creadas en tareas anteriores. La función \script{create\_matrix\_system} aplica la ecuación \ref{eq:heat_system_finish} para obtener la ecuación matricial.