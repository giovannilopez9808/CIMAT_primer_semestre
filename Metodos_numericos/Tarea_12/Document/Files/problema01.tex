\section*{Problema 1}
\textbf{Programar el algoritmo QR para encontrar los eigenpares de las matrices \file{Eigen\_3.txt} y \file{Eigen\_25.txt}. Únicamente reportar: criterio de paro establecido, tolerancia utilizada, iteraciones requeridas, autovalores aproximados y la comprobación dada por}
\begin{equation*}
	\frac{||AV-\Lambda V||_2}{||AV||_2}
\end{equation*}

\textbf{donde A es la matriz en cuestión. V y $\Lambda$ son, respectivamente, las matrices de eigenvectores y eigenvalores aproximados.}

\subsection*{Criterio de paro}

La condición establecida en el algoritmo implementado se basa en que el resultado de la multiplicación de las matrices RQ es simetrica. Entonces, se encuentra el valor máximo en la parte superior de la matriz. La tolerancia usada fue de 10$^{-6}$. Para obtener el eigenvalor más grande se implemento el método de la potencia usando la misma tolerancia que el algoritmo QR. Esto debido a que la definción de la norma 2 de  A es la siguiente:

\begin{equation*}
	||A||_2 = max_\lambda{A^TA}
\end{equation*}

\subsection*{Iteraciones requeridas y comprobación}

Las iteraciones realizadas y la comprobación de cada matriz se encuentran la tabla \ref{table:iteraciones}.

\begin{table}[H]
	\centering
	\begin{tabular}{lcc}\hline
		Matriz               & Iteraciones & Comprobación \\  \hline
		\file{Eigen\_3.txt}  & 27          & 1.052200     \\
		\file{Eigen\_25.txt} & 622         & 0.998071     \\ \hline
	\end{tabular}
	\caption{Iteraciones realizadas y comprobaciones usando el algoritmo QR}
	\label{table:iteraciones}
\end{table}

\pagebreak
