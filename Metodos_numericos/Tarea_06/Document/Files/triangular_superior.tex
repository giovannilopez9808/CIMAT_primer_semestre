\subsection{Matriz triangular superior}

Suponiendo que se tiene un sistema de ecuaciones de n ecuaciones donde los coeficientes son
iguales a cero cuando la posición del coeficiente es menor a el número de la ecuación. Entonces representando el sistema de ecuaciones antes descrito en una ecuación matricial se obtiene lo siguiente:

\begin{equation*}
    \begin{pmatrix}
        a_{11} & a_{12} & \cdots & a_{1n} \\
        0      & a_{22} & \cdots & a_{2n} \\
        \vdots & \vdots & \ddots & \vdots \\
        0      & 0      & \cdots & a_{nn}
    \end{pmatrix}
    \begin{pmatrix}
        x_1    \\
        x_2    \\
        \vdots \\
        x_n
    \end{pmatrix} =
    \begin{pmatrix}
        b_1    \\
        b_2    \\
        \vdots \\
        b_n
    \end{pmatrix}
\end{equation*}

realizando la multiplicación de matrices de obtiene que:

\begin{equation*}
    \begin{pmatrix}
        a_{11}x_{1}+a_{12}x_{2}+\cdots+a_{1n}x_n \\
        a_{22}x_{2}+a_{23}x_{3}+\cdots+a_{2n}x_n \\
        \vdots                                   \\
        a_{nn}x_{n}
    \end{pmatrix} =
    \begin{pmatrix}
        b_1    \\
        b_2    \\
        \vdots \\
        b_n    \\
    \end{pmatrix}
\end{equation*}

reduciendo esto a una expresión generalizada se obtiene que la solución de la i-ésima variable es:

\begin{equation*}
    x_i = \frac{b_{i}-\sum\limits_{j=i+1}^n a_{ij}x_{j}}{a_{ii}}
\end{equation*}

Se dice que el sistema de ecuaciones no tiene solución cuando algún elemento de la diagonal $(a_{ii})$ es igual a cero. El algoritmo planteado para realizar la solución a este tipo de ecuacioes es el siguiente:

\begin{lstlisting}[style=CStyle]
    // input: matriz, vector_b
    // output: solutions
    for(i = n; i <= 1; i--)
    {
        sum_i = 0
        valid_solution(matriz[i][i])
        for(j = n; j <= i+1; j--)
        {
            sum_i += matriz[i][j]*solutions[j]
        }
        solutions[i] = (vector_b[i] - sum_i) / matriz[i][i]
    }

\end{lstlisting}

La imprementación de este problema se encuentra en la carpeta \folder{Problema\_2}. La función \script{solve\_triangular\_superior\_matrix} esta contenido en el archivo \file{solution.h}