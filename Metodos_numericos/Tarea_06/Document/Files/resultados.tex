\section{Resultados}

\subsection{Matriz diagonal}

El sistema de ecuaciones planteado en los archivos de prueba (\file{M\_DIAG.txt} y \file{V\_DIAG.txt}) corresponde a la siguiente ecuación matricial:

\begin{equation*}
    \begin{pmatrix}
        1 & 0 & 0 & 0 \\
        0 & 2 & 0 & 0 \\
        0 & 0 & 3 & 0 \\
        0 & 0 & 0 & 4 \\
    \end{pmatrix}
    \begin{pmatrix}
        x_1 \\
        x_2 \\
        x_3 \\
        x_4
    \end{pmatrix} = \begin{pmatrix}
        1 \\
        2 \\
        3 \\
        4
    \end{pmatrix}
\end{equation*}

Ejecutando el programa escrito, se obtiene los siguientes resultados:

\begin{lstlisting}[language=bash]
    --------------------------------------------
    Soluciones del sistema

    x_1	= 1.000000
    x_2	= 1.000000
    x_3	= 1.000000
    x_4	= 1.000000

\end{lstlisting}

\subsection{Matriz triangular superior}

El sistema de ecuaciones planteado en los archivos de prueba (\file{M\_TSUP.txt} y \file{V\_TSUP.txt}) corresponde a la siguiente ecuación matricial:

\begin{equation*}
    \begin{pmatrix}
        1 & 2 & 3 & 4  \\
        0 & 5 & 6 & 7  \\
        0 & 0 & 8 & 9  \\
        0 & 0 & 0 & 10 \\
    \end{pmatrix}
    \begin{pmatrix}
        x_1 \\
        x_2 \\
        x_3 \\
        x_4
    \end{pmatrix} = \begin{pmatrix}
        1 \\
        2 \\
        3 \\
        4
    \end{pmatrix}
\end{equation*}

Ejecutando el programa escrito, se obtiene los siguientes resultados:

\begin{lstlisting}[language=bash]
    --------------------------------------------
    Soluciones del sistema
    
    x_1	= -0.141667
    x_2	= -0.116667
    x_3	= -0.075000
    x_4	= 0.400000
    

\end{lstlisting}

\subsection{Matriz triangular inferior}

El sistema de ecuaciones planteado en los archivos de prueba (\file{M\_TINF.txt} y \file{V\_TINF.txt}) corresponde a la siguiente ecuación matricial:

\begin{equation*}
    \begin{pmatrix}
        1 & 0 & 0 & 0  \\
        2 & 3 & 0 & 0  \\
        4 & 5 & 6 & 0  \\
        7 & 8 & 9 & 10 \\
    \end{pmatrix}
    \begin{pmatrix}
        x_1 \\
        x_2 \\
        x_3 \\
        x_4
    \end{pmatrix} = \begin{pmatrix}
        1 \\
        2 \\
        3 \\
        4
    \end{pmatrix}
\end{equation*}

Ejecutando el programa escrito, se obtiene los siguientes resultados:

\begin{lstlisting}[language=bash]
    --------------------------------------------
    Soluciones del sistema
    
    x_1	= 1.000000
    x_2	= 0.000000
    x_3	= -0.166667
    x_4	= -0.150000

\end{lstlisting}

\subsection{Matriz Eliminación Gaussiana}


La solución del sistema de ecuaciones planteado en los archivos de prueba (\file{M\_LARGE.txt} y \file{V\_LARGE.txt}) es extenso para ser escrito en este reporte, es por ello que se encuentra en el archivo \file{Solution\_LARGE.txt}, el cual esta contenido en la carpeta \folder{Problema\_4}.

Se probo con otro sistema de ecuaciones de menor tamaño para comprobar su solución. Este se encuentra en los archivos \file{test\_matrix.txt} y \file{test\_result.txt}, el cual es el siguiente:

\begin{equation*}
    \begin{pmatrix}
        2 & 1  & -3 \\
        5 & -4 & 1  \\
        1 & -1 & -4 \\
    \end{pmatrix}
    \begin{pmatrix}
        x_1 \\
        x_2 \\
        x_3 \\
        x_4
    \end{pmatrix} = \begin{pmatrix}
        7   \\
        -19 \\
        4
    \end{pmatrix}
\end{equation*}

Ejecutando el programa escrito, se obtiene los siguientes resultados:

\begin{lstlisting}[language=bash]
    --------------------------------------------
    Soluciones del sistema
    
    x_1	= -1.000000
    x_2	= 3.000000
    x_3	= -2.000000    

\end{lstlisting}