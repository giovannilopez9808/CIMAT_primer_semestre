\item \textbf{Tomas un mazo de cartas. Solamente te fijas en el valor númerico correspondiente (campesino=11,dama=12, rey=13). Eliges una carta al azar, apuntas su valor numérica y la regresas. Eligas otra carta y también apuntas su valor numérico. ¿Cuál es la probabilidad que el primer valor sea mayor que el segundo?}

El total de permutaciones posibles en este caso son 9, de las cuales 3 se obtiene primero un valor mayor al segundo. Esto es representado en la tabla \ref{table:problema5}.
\begin{table}
    \centering
    \begin{tabular}{|c|c|c|} \hline
        13 13                           & 12 13                           & 11 13 \\ \hline
        \textbf{\textcolor{red}{13 12}} & 12 12                           & 11 12 \\ \hline
        \textbf{\textcolor{red}{13 11}} & \textbf{\textcolor{red}{12 11}} & 11 11 \\ \hline
    \end{tabular}
    \caption{Total de opciones posibles del problema 5. Pares de números que cumplen la condición del problema (color rojo).}
    \label{table:problema5}
\end{table}
Por lo tanto, la probabilidad que suceda este evento (A) es
\begin{equation*}
    P(A)= \frac{1}{3}
\end{equation*}