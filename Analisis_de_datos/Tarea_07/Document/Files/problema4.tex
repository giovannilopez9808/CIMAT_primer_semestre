\section*{Problema 4}

\textbf{El tiempo de ejecucción de un programa sigue una distribución normal. Para una muestra de tamaño 40 se obtiene que $\bar{x}=32.2$s y $\sigma^2=3.1\; s^2$.}

\begin{enumerate}
    \item \textbf{¿Cuantas veces se debe ejecutar el programa para obtener un intervaloc de confianza de 95\% con un ancho menor que 2 s?}

          Se tiene que un intervalo de confianza es:

          \begin{equation*}
              \left [ \frac{1}{n} \sum x_i - \frac{Z_{\alpha/2}\sigma}{\sqrt{n}},\frac{1}{n} \sum x_i + \frac{Z_{\alpha/2}\sigma}{\sqrt{n}} \right ]
          \end{equation*}

          entonces, el ancho del intervalo es:

          \begin{equation*}
              d = \frac{2Z_{\alpha/2}\sigma}{\sqrt{n}}
          \end{equation*}

          por ende:

          \begin{equation*}
              n = \frac{4Z_{\alpha/2}^2\sigma^2}{d^2}
          \end{equation*}

          por lo tanto:

          \begin{align*}
              n & = \frac{4(1.959964)^2(3.1)^2}{2^2} \\
              n & = 36.9164
          \end{align*}

    \item \textbf{En muchas situaciones, el interés no es tanto en el comportamiento promedio, si no en la variabilidad. Usando el hecho que para una muestra de una misma distribución normal con varianza $\sigma^2$}
          \begin{equation*}
              \frac{(n-1)S^2}{\sigma^2} \sim \chi_{n-1}^2
          \end{equation*}

          \textbf{con $S^2 = \frac{1}{n}\sum (X_i-\bar{X})^2$ y $\chi_{n-1}^2$ una distribución chicuadrada, deriva un intervalo de confianza de 95\% para la varianza.}

          Se tiene que

          \begin{equation*}
              P(\chi_{n-1,\alpha/2}^2\leq \frac{(n-1)S^2}{\sigma^2}\leq \chi_{n-1,1-\alpha/2}^2) = 1-\alpha
          \end{equation*}


          entonces:

          \begin{equation*}
              P\left (\frac{(n-1)S^2}{\chi_{n-1,\alpha/2}} \leq \sigma^2 \leq \frac{(n-1)S^2}{\chi_{n-1,1-\alpha/2}} \right )
          \end{equation*}

          por lo tanto, el intervalo de confianza para la varianza es:

          \begin{equation*}
              \left (\frac{(n-1)S^2}{\chi_{n-1,\alpha/2}},\frac{(n-1)S^2}{\chi_{n-1,1-\alpha/2}}\right )
          \end{equation*}

          Para un intervalo de confianza del 95\%, se tiene $\alpha = 0.05$. El intervalo de confianza para la varianza es:

          \begin{equation*}
              \left (\frac{(n-1)S^2}{\chi_{n-1,0.025}},\frac{(n-1)S^2}{\chi_{n-1,0.975}}\right )
          \end{equation*}
\end{enumerate}