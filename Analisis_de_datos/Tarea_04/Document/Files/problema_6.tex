\section*{Problema 6}

\textbf{Se tiene una barra de longitud 1 metro y una posición p dada en la barra. Se rompe la barra en dos en un punto elegido al azar. Calcula la longitud promedio de la pieza de la barra que contiene p $(0 < p < 1)$.}

Sea A todos los puntos posibles de elección.

\begin{equation*}
    A=\{p \in [0,1] \}
\end{equation*}

Se tiene que para un punto p escogido de la barra, se obtienen las siguientes longitudes:

\begin{equation*}
    L(p) = \begin{cases}
        x   & \text{para }  0<p<x \\
        1-x & \text{para }  x<p<1
    \end{cases}
\end{equation*}

donde x es el punto donde se partio la barra. Entonces, el promedio de las logitudes qe contienen a p es:

\begin{align*}
    E(L(p)) & = \int_A xL(p)dx                                                                               \\
            & = \int_0^p x(x)dx + \int_p^1 x(1-x)dx                                                          \\
            & = \left. \frac{x^2}{2}\right|_0^p + \left. \left(\frac{x^2}{2}-\frac{x^3}{3}\right)\right|_p^1 \\
            & = \frac{p^2}{2} + \frac{1}{2} - \frac{1}{3} - \frac{p^2}{2} + \frac{p^3}{3}                    \\
    E(L(p)) & = \frac{p^3}{3} + \frac{1}{6}
\end{align*}