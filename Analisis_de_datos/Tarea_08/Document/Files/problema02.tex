\section*{Problema 2}

\textbf{Para investigar si una moneda es justa o no, alguien decide que va a lanzar la moneda 100 veces. Si el número de veces de obtener sol es entre 42\% y 58\%, va a apoyar la hipótesis de que la moneda es justa. Calcula el nivel de significancia correspondiente.}


Sea X una variable aleatoria tal que $X\sim Bern(0.5)$, entonces se tiene que $\mu=0.5$ y $\sigma^2 = 0.25$. Entonces el valor de  t para la distribución t-student es:

\begin{equation*}
    t_{100,\alpha} = \frac{0.58-\mu}{\frac{\sigma}{\sqrt{n}}}
\end{equation*}

entonces

\begin{align*}
    t_{100,\alpha} & = \frac{0.58-0.5}{\frac{0.5}{\sqrt{100}}} \\
                   & = \frac{0.08}{0.05}                       \\
    t_{100,\alpha} & = 1.6
\end{align*}

por lo tanto usando la función \script{pt} con $t_{100,\alpha}$ se obtiene que el nivel de significancia es $\alpha=0.05639257$.