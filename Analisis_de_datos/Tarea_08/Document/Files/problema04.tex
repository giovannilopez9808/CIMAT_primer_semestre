\section*{Problema 4}

\textbf{Considera los siguientes datos de un estudio en Bélgica sobre la intención de voto entre 1000 parejas. Las variables $X_1$ , $X_2$ indican si la mujer, respectivamente el hombre, votará para un partido de la coalición (0) o de la oposición (1) en caso de que hubieran elecciones en ese momento.}

\begin{table}[H]
	\centering
	\begin{tabular}{c|cc}
		        & $X_1=0$ & $X_1=1$ \\  \hline
		$X_2=0$ & 245     & 170     \\
		$X_2=1$ & 218     & 367
	\end{tabular}
\end{table}

\begin{enumerate}
	\item Calcula el oddsratio $\hat{R}$
	\item Se puede mostrar que si el tamaño de la muestra va a $\infty$, la distribución de log($\hat{R}$) converge a una normal con promedio log($\hat{R}$), el verdadero log-oddsratio de la distribución subyacente, y con varianza

	      \begin{equation*}
		      \frac{1}{n_{0,0}} + \frac{1}{n_{0,1}} + \frac{1}{n_{1,0}} + \frac{1}{n_{1,1}}
	      \end{equation*}

	      donde $n_{i,j}$ es el número de  observaciones con $X_1=i$ y $X_2=j$. ¿Apoyas la hipótesis que la pareja vota de manera independiente ($\alpha=0.05$)?
\end{enumerate}
