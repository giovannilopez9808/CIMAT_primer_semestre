\section*{Problema 3}

\textbf{Por experiencia se sabe que el número de  accesos ,X, durante una hora a una base de datos sigue una distribución Poisson:}

\begin{equation*}
	P(X=x) = exp(-\lambda) \frac{\lambda^x}{x!} \qquad \text{para } x=0,1,2,\dots \; \text{y } \lambda>0
\end{equation*}

\textbf{Calcula el estimador de máxima verosimilitud para $\lambda$ en una muestra.}

La función de máximo verosimilitud de la distribución es Poisson es:

\begin{equation*}
	\mathcal{L} = \prod_{i=1}^n e^{-\lambda} \frac{\lambda^x}{x!}
\end{equation*}

Entonces, la función log-verosimilitud es:

\begin{align*}
	log(\mathcal{L}) & = log\left (\prod_{i=1}^n e^{-\lambda} \frac{\lambda^x}{x!}\right )            \\
	                 & = \sum_{i=1}^n log \left ( e^{-\lambda} \frac{\lambda^x}{x!}\right )           \\
	                 & = \sum_{i=1}^n \left [ log(e^{-\lambda})+ log(\lambda^{x_i}) -log(x!) \right ] \\
	                 & = \sum_{i=1}^n \left [ -\lambda +x_i log(\lambda) -log(x_i)\right ]            \\
	log(\mathcal{L}) & = -\lambda n + log(\lambda) \sum x_i - \sum log(x_i!)
\end{align*}

Por lo que calculando el valor crítico de $log(\mathcal{L})$ se obtiene que:

\begin{align*}
	\frac{\partial \log(\mathcal{L})}{\partial \lambda} & = 0                    \\
	-n  + \frac{\sum x_i}{\lambda}                      & = 0                    \\
	\hat{\lambda}                                       & = \frac{1}{n} \sum x_i
\end{align*}

donde $\hat{\lambda}$ corresponde a la media aritmética.

Calculando la segunda derivada de $log(\mathcal{L})$ para comprobar que el valor crítico ($\hat{\lambda}$) se trata de un máximo se obtiene lo siguiente:

\begin{align*}
	\frac{\partial^2 \log(\mathcal{L})}{\partial \lambda^2} & < 0 \\
	-\frac{1}{\lambda^2} \sum x_i  < 0
\end{align*}

como $\lambda^2 > 0$ y $x_i \in \{0,1,2,\dots\}$, entonces, la segunda derivada de $log(\mathcal{L})$ será siempre negativa, por lo tanto $log(\mathcal{L})$ es el estimador de máxima verosimilitud.
