\subsection*{Problema 4}
\begin{itemize}
    \item \textbf{Si A y B son independientes, A y B\textsuperscript{c} son independientes. Demuéstralo.}

          Se tiene que:
          \begin{equation}
              P(B^c|A)  =1-P(B|A)
              \label{eq:pbc_a}
          \end{equation}
          como A y B son independientes se cumple que:
          \begin{equation*}
              P(B|A)= P(B)
          \end{equation*}
          entonces, la ecuación \ref{eq:pbc_a} puede reescribirse como:
          \begin{align*}
              P(B^c|A) & =1-P(B|A) \\
                       & = 1- P(B) \\
                       & = P(B^c)
          \end{align*}
          por lo tanto
          \begin{equation}
              P(B^c|A) = P(B^c)
              \label{eq:result4a}
          \end{equation}
          La ecuación \ref{eq:result4a} es valida unicamente si los eventos $B^c$ y A son independientes.
    \item \textbf{Cierto o falso: si A y B son independientes, A\textsuperscript{c} y B\textsuperscript{c} son independientes. Demuéstralo}

          Calculando $P(A^c|B^c)$
          \begin{equation}
              P(A^c|B^c) = 1-P(A|B^c)
              \label{pac_bc}
          \end{equation}
          como en el ejercicio anterior se demostro que si A y B son independientes entonces A y $B^c$ son independientes, se tiene que
          \begin{equation}
              P(A|B^c) = P(A)
              \label{eq:pa_bc}
          \end{equation}
          Usando la ecuación \ref{eq:pa_bc} en la ecuación \ref{pac_bc} se obtiene que:
          \begin{align*}
              P(A^c|B^c) & = 1-P(A|B^c) \\
                         & = 1-P(A)     \\
                         & =P(A)
          \end{align*}
          por lo tanto
          \begin{equation}
              P(A^c|B^c)=P(A)
              \label{eq:result4b}
          \end{equation}
          La ecuación \ref{eq:result4b} es valido unicamente si $A^c$ y $B^c$ son independientes.
\end{itemize}