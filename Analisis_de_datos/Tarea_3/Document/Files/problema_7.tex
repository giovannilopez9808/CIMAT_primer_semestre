\section*{Problema 7}
\textbf{Se debe sujetar N personas a una prueba de sangre para detectar la posible presencia de una cierta enfermedad. Con ese fin se dividen las personas al azar en subgrupos de tamaño k (puedes suponer que N es un múltiple de k). Se toma una muestra de sangre de cada persona y se mezclan las que pertenecen a personas de un mismo subgrupo; se aplica la prueba a estas k mezclas. Si el resultado es positivo, se sujeta cada persona del subgrupo correspondiente a una prueba separada. Suponiendo que la probabilidad de tener la enfermedad es 0.01, y que la presencia de la enfermedad entre las personas ocurre de manera independiente: calcula el promedio del número de pruebas que se va a tener que aplicar.}