\section*{Problema 8}
\subsection*{Problema 8a}
\textbf{Elige al azar sin remplazo n números de \{1, ..., n\}. Da un argumento porque la probabilidad que el último obtenido sea el k-ésimo mayor es igual a $\frac{1}{n}$(1 $\leq$ k $\leq$ n).}

Sea
\begin{equation*}
    \Omega = \{ \Omega_i : 1,2,\dots,n \}
\end{equation*}

Y sea $A\subset \Omega$ tal que $\omega_k \notin A$ donde $\omega_k > \omega_i$. Entonces en el primer intento se tiene que la probabilidad de obtener un número contenido en A es:
\begin{equation*}
    P(A_1) = \frac{n-1}{n}
\end{equation*}
para el segundo intento es
\begin{equation*}
    P(A_2) = \frac{n-2}{n-1}
\end{equation*}

si realizamos esta sucesion para obtener todos los números contenidos en A se tiene que:

\begin{align*}
    P(A_1,A_2,\dots,A_{n-1}) & = P(A_1)P(A_2)\dots P(A_{n-1})                                                          \\
                             & = \left(\frac{n-1}{n}\right)\left(\frac{n-2}{n-1}\right) \dots \left(\frac{1}{2}\right) \\
                             & = \frac{1}{n}
\end{align*}


Como el único número que no ha sido obtenido es $\Omega_k$, entonces, la probabilidad de obtenerlo es $\frac{1}{n}$.
\subsection*{Problema 8b}
\textbf{Considera el siguiente código para encontrar el máximo en un arreglo
    de n números enteros positivos (todos diferentes).}
\begin{lstlisting}[language=python]
    max=-1
    for (i in 1:n)
        {
        if (A[i] > max)
            max = A[i]
        }
\end{lstlisting}
\textbf{Supongamos que el orden de los elementos es totalmente al azar. Define X el número de veces que se actualiza la variable max. Calcula EX (usa el inciso anterior).}