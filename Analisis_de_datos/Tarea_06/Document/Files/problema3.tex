\section*{Problema 3}

\textbf{Supongamos que se quiere estimar el número promedio
    $\mu$ de amigos que alguien tiene en Facebook. Se toma una muestra de
    personas y ellos eligen al azar algunos de sus amigos en Facebook. Se
    calcula el promedio del número de amigos que estos amigos tienen. Aunque
    suponemos independencia, argumenta que en general se va a sobrestimar
    $\mu$ de esta manera.}

El número de personas que tiene alguien en Facebook puede depender de la edad, actividades que haga. Entonces puede llegar a crear sesgos. Esto es debido a que se pueden tener amigos mutuos o coincidir en varios grupos. Por ende tener un número de amigos semejante, dando así un promedio de amigos sobrestimado.