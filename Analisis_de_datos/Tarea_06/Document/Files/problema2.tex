\section*{Problema 2}

\textbf{Sea X=(X\textsubscript{1},X\textsubscript{2},X\textsubscript{3})$\sim \mathcal{N}(\mu , \Sigma)$ con}
\begin{equation*}
    \mu^T = (2,-3,1) \qquad \Sigma = \begin{pmatrix}
        1 & 1 & 1 \\
        1 & 3 & 2 \\
        1 & 2 & 2 \\
    \end{pmatrix}
\end{equation*}
\begin{enumerate}
    \item \textbf{Encuentra la distribución de X\textsubscript{1}+X\textsubscript{2}-X\textsubscript{3}.}

          Sea $X_i \sim \mathcal{N}(\mu_i,\Sigma_{ii})$, entonces se tiene que:

          \begin{equation*}
              Y = \sum_{i=1}^3 a_i X_i
          \end{equation*}

          la cual es una combinación lineal de $X_1,X_2,X_3$ donde $a=\{1,1-1\}$. Entonces $Y \sim N(EY,Var(T))$. Calculando $EY$, se tiene lo siguiente:

          \begin{align*}
              EY & = E(X_1 + X_2 -X_3)  \\
                 & = EX_1 + EX_2 - EX_3 \\
                 & = 2 -3 -1            \\
              EY & = -2
          \end{align*}

          Calculando $Var(Y)$ se tiene lo siguiente:

          \begin{align*}
              Var(Y) & = Var(X_1 + X_2 - X_3)                                                                \\
                     & =Var(X_1) + Var(X_2) + (-1)^2Var(X_3) +2Cov(X_1,X_2) +2Cov(X_2,-X_3) + 2Cov(X_1,-X_3) \\
                     & =Var(X_1) + Var(X_2) + Var(X_3) +2Cov(X_1,X_2) -2Cov(X_2,X_3) - 2Cov(X_1,X_3)         \\
                     & = 1 +2 +3 + 2(1) - 2(2)-2(1)                                                          \\
              Var(Y) & =   1
          \end{align*}

          Entonces la distribución $X_1+X_2-X_3 \sim \mathcal{N}(-2,1)$.

    \item \textbf{Calcula EX\textsubscript{1}$|$X\textsubscript{2}=2}

          Definamos a $X_1$ y $X_2$ como lo siguiente:

          \begin{equation*}
              \begin{cases}
                  X_1 & = \mu_1 + \sigma_1 \mathcal{Z}_1                                                      \\
                  X_2 & = \mu_2 + \sigma_2 \left ( \rho \mathcal{Z}_1  + \sqrt{1-\rho^2}\mathcal{Z}_2\right )
              \end{cases}
          \end{equation*}

          donde $\mathcal{Z}_1 , \mathcal{Z}_2 \sim \mathcal{N}(0,1)$.

          Entonces calculando $EX_2|X_1=x$ se tiene lo siguiente:

          \begin{align*}
              E(X_2 | X_1=x) & = E(\mu_2 + \sigma_2 \left ( \rho \mathcal{Z}_1  + \sqrt{1-\rho^2}\mathcal{Z}_2\right )|X_1=x)
          \end{align*}

          por linealidad de la esperanza se obtiene lo siguiente:

          \begin{align*}
              E(X_2 | X_1=x) & =E(\mu_2|X_1=x) + E(\sigma_2 \rho \mathcal{Z}_1|X_1=x) + E(\rho \sqrt{1-\rho^2}\mathcal{Z}_1|X_1=x)
          \end{align*}

          como $\mathcal{Z}_1$ y $\mathcal{Z}_2$ son distribuciones independientes de x entonces:

          \begin{align*}
              E(X_2 | X_1=x) & =\mu_2 + \sigma_2 \rho \mathcal{Z}_1 + \sigma_2 \sqrt{1-\rho^2}E(\mathcal{Z}_2)
          \end{align*}

          como $E\mathcal{Z}_2=0$ entonces:

          \begin{align*}
              E(X_2 | X_1=x) & =\mu_2 + \sigma_2 \rho \mathcal{Z}_1               \\
              E(X_2 | X_1=x) & = \mu_2 + \frac{\sigma_2}{\sigma_1} \rho (x-\mu_1)
          \end{align*}

          Para este caso se tiene que $\rho = \frac{\sqrt{3}}{3}, \mu_1=2, \mu_2 = -3,\sigma_1=1 ,\sigma_2 = \sqrt{3}$, entonces:

          \begin{align*}
              E(X_2 | X_1=2) & = \mu_2 + \frac{\sigma_2}{\sigma_1} \rho (x-\mu_1)              \\
                             & = -3 + \frac{\sqrt{3}}{1}\left (\frac{\sqrt{3}}{3}\right )(2-2) \\
              E(X_2 | X_1=2) & = -3
          \end{align*}

          Observando el resultado uno podria llegar a equivocarse y decir que $X_2$ y $X_1$ son independientes, pero esto es una coincidendia condiconar a la variable $X_1$ con su promedio.

    \item \textbf{Encuentra un vector $\mathbf{v}$ tal que X\textsubscript{2} y X\textsubscript{2}-$\mathbf{v^T \begin{pmatrix} X_1 \\ X_3 \end{pmatrix}}$ sean independientes.}

          Sea $Y=X_2 + v^T \begin{pmatrix}
                  X_1 \\ X_3
              \end{pmatrix}$, entonces, esta variable es igual a:

          \begin{align*}
              Y & = X_2 + v^T \begin{pmatrix}
                  X_1 \\ X_3
              \end{pmatrix}                        \\
                & = X_2 + \begin{pmatrix}
                  a & b
              \end{pmatrix} \begin{pmatrix}
                  X_1 \\ X_3
              \end{pmatrix} \\
              Y & = X_2 -aX_1 -b X_2
          \end{align*}

          Se tiene que si $X_2$ y $Y$ son independientes, entonces:

          \begin{equation*}
              EX_2Y = EX_2 EY
          \end{equation*}

          Calculando $EX_2Y$ se tiene que:

          \begin{align*}
              E(X_2Y) & = E(X_2^2-aX_1X_2-bX_2X_3)       \\
                      & = EX_2^2+E(-aX_1X_2)+E(-bX_2X_3) \\
                      & = EX_2^2-aEX_1X_2-bEX_2X_3       \\
          \end{align*}

          como las variables $X_i$ son independientes entre si, entonces:

          \begin{equation}
              EX_2Y = EX_2^2-aEX_1EX_2-bEX_1EX_3 \label{eq:EX2Y}
          \end{equation}

          Calculando $EX_2EY$ se tiene que:


          \begin{align*}
              EX_2EY & = EX_2 E(X_2-aX_1-bX_3)           \\
                     & =EX_2( EX_2-aEX_1-bEX_3)          \\
              EX_2EY & = (EX_2)^2 -aEX_1EX_2 - bEX_2EX_3
          \end{align*}

          entonces

          \begin{equation}
              EX_2EY  = (EX_2)^2 -aEX_1EX_2 - bEX_2EX_3 \label{eq:EX2EY}
          \end{equation}

          Igualando las ecuacione \ref{eq:EX2Y} y \ref{eq:EX2EY} se tiene lo siguiente:

          \begin{align*}
              EX_2Y                     - EX_2EY                          & =0  \\
              EX_2^2-aEX_1EX_2-bEX_1EX_3 -(EX_2)^2 +aEX_1EX_2 + bEX_2EX_3 & = 0 \\
              EX_2^2 -(EX_2)^2                                            & =0  \\
              Var(X_2)                                                    & =0
          \end{align*}

          entonces, el vector $v$ para que $X_2$ y $Y$ sean independientes puede ser cualquiera, siempre y cuando $Var(X_2)=0$. Al ser $X_2 \sim \mathcal{N}(\mu,\sigma)$, la cual requiere que $\sigma>0$, entonces no existe un vector $v$.
\end{enumerate}