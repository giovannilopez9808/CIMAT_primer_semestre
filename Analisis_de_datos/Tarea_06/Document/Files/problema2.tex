\section*{Problema 2}

\textbf{Sea X=(X\textsubscript{1},X\textsubscript{2},X\textsubscript{3})$\sim \mathcal{N}(\mu , \Sigma)$ con}
\begin{equation*}
    \mu^T = (2,-3,1) \qquad \Sigma = \begin{pmatrix}
        1 & 1 & 1 \\
        1 & 3 & 2 \\
        1 & 2 & 2 \\
    \end{pmatrix}
\end{equation*}
\begin{enumerate}
    \item \textbf{Encuentra la distribución de X\textsubscript{1}+X\textsubscript{2}-X\textsubscript{3}.}

          Sea $X_i \sim \mathcal{N}(\mu_i,\Sigma_{ii})$, entonces se tiene que:

          \begin{equation*}
              Y = \sum_{i=1}^3 a_i X_i
          \end{equation*}

          la cual es una combinación lineal de $X_1,X_2,X_3$ donde $a=\{1,1-1\}$. Entonces $Y \sim N(EY,Var(T))$. Calculando $EY$, se tiene lo siguiente:

          \begin{align*}
              EY & = E(X_1 + X_2 -X_3)  \\
                 & = EX_1 + EX_2 - EX_3 \\
                 & = 2 -3 -1            \\
              EY & = -2
          \end{align*}

          Calculando $Var(Y)$ se tiene lo siguiente:

          \begin{align*}
              Var(Y) & = Var(X_1 + X_2 - X_3)                                                              \\
                     & Var(X_1) + Var(X_2) + (-1)^2Var(X_3) +2Cov(X_1,X_2) +2Cov(X_2,-X_3) + 2Cov(X_1,X_3)
          \end{align*}
    \item \textbf{Calcula EX\textsubscript{1}$|$X\textsubscript{2}=2}
    \item \textbf{Encuentra un vector $\mathbf{v}$ tal que X\textsubscript{2} y X\textsubscript{2}-$\mathbf{v^T \begin{pmatrix} X_1 \\ X_3 \end{pmatrix}}$ sean independientes.}
\end{enumerate}