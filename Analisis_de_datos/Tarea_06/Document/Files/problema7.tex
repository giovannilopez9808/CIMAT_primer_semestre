\section*{Problema 7}

\textbf{Considera el siguiente método para estimar el tamaño (N) de una población de animales de un especie particular. Primero se capturan M animales, los marcan y son puestos de nuevo en libertad. Un tiempo más tarde se capturan animales hasta encontrar un animal marcado. Sea X el número total de animales capturados (X incluye el animal marcado). Después se dejan todos los animales en libertad. Se repite lo anterior de tal forma que se obtenga una muestra $\{x_1,x_2,\dots,x_n\}$ de X (así este procedimiento puede tardar bastante). Puedes suponer que en cada momento la probabilidad de capturar un animal marcado es siempre igual (así se supone que N es mucho mayor que M ).
}

\begin{enumerate}
    \item \textbf{Demuestre que:}
          \begin{equation*}
              P(X=x) = \frac{M}{N} \left (1-\frac{M}{N}\right )^{x-1}, \qquad x=1,2,\dots
          \end{equation*}
    \item \textbf{Demuestra que:}
          \begin{equation*}
              \hat{\Theta}_n = \frac{M}{N} \sum_{i=1}^n X_i
          \end{equation*}
          \textbf{es el estimador de Máximo verosimilitud. ¿Está insesgado?¿Qué puedes decir si $\mathcal{n\rightarrow \infty}$?}
\end{enumerate}