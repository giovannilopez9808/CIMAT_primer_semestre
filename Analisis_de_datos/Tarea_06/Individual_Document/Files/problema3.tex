\section*{Problema 3}

\textbf{Supongamos que se quiere estimar el número promedio
    $\mu$ de amigos que alguien tiene en Facebook. Se toma una muestra de
    personas y ellos eligen al azar algunos de sus amigos en Facebook. Se
    calcula el promedio del número de amigos que estos amigos tienen. Aunque
    suponemos independencia, argumenta que en general se va a sobrestimar
    $\mu$ de esta manera.}

El número de personas que tiene alguien en Facebook depende de la edad, sus circulos de conocidos y localización de donde habita. Estas variables se pueden tomar en cuenta un grupo inicial de personas, pero se pueden generar sesgos al escoger amigos de una sola persona. Esto debido a que pueden tener amigos mutuos o coincidir en varios grupos sociales. Dando como resultado un número de amigos semejantes.