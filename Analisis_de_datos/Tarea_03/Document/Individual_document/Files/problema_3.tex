\section*{Problema 3}
\textbf{Sea X una v.a. con la siguiente distribución. Calcula Var(X$^2$) y E(X$^2$ $|$ X$>$1)}
\begin{table}[H]
    \centering
    \begin{tabular}{l|cccc}
               & 1   & 2   & 3   & 4   \\ \hline
        P(X=x) & 0.2 & 0.1 & 0.4 & 0.3
    \end{tabular}
\end{table}

Se tiene que:
\begin{equation}
    Var(X^2) = E(X^4)- (E(X^2))^2
\end{equation}

Calculando $E(X^2)$, se tiene que:

\begin{align*}
    E(X^2) & =  \sum_i x_i^2 P(x_i)                       \\
           & = 1^2(0.2) + 2^2(0.1) + 3^2 (0.4) + 4^2(0.3) \\
           & = 9
\end{align*}

Calculando $E(X^4)$, se tiene que:

\begin{align*}
    E(X^4) & =  \sum_i x_i^2 P(x_i)                       \\
           & = 1^4(0.2) + 2^4(0.1) + 3^4 (0.4) + 4^4(0.3) \\
           & = 111
\end{align*}

Entonces:
\begin{align*}
    Var(X^2) & = 111 - 9^2 \\
             & = 111-81    \\
             & =30
\end{align*}

por lo tanto, para la distribución dada se tiene:
\begin{align*}
    Var(X^2) = 30
\end{align*}


Definimos al conjunto A como el siguiente:

\begin{equation}
    A = \{x : x>1 \}
\end{equation}

Usando la distribución dada, entonces el conjunto A esta constituido de la siguiente manera:

\begin{equation*}
    A = \{2,3,4 \}
\end{equation*}

Calculando $E(X^2|X>1)$, se obtiene lo siguiente:

\begin{equation}
    E(X^2|X>1) = \sum_i x_i^2 P(X=x_i | X >1 )
    \label{eq:definition_ex2x1}
\end{equation}

Si $x_i \in A$, entonces se obtiene lo siguiente:

\begin{align*}
    P(X=x_i | X>1) & = \frac{P(x_i,X>1)}{P(X>1)} \\
                   & = \frac{P(x_i)}{P(X>1)}
\end{align*}

Esto es porque la interseccion de $x_i$ con A es el mismo $x_i$. En el caso contrario $x_i \notin A$, entonces $P(X=x_i | X>1) =0$, ya que no existira intersección entre $x_i$ y A. Por lo tanto:

\begin{equation}
    P(X=x_i | X>1) = \left\lbrace \begin{matrix}
        0                     & \text{para} & x_i\notin A \\
        \frac{P(X=x_i)}{P(A)} & \text{para} & x_i\in A
    \end{matrix} \right. \label{eq:problema3_pxix1}
\end{equation}

Calculando $P(X>1)$, se obtiene lo siguiente:

\begin{align}
    P(X>1) & = P(X=2) + P(X=3) + P(X=4) \nonumber \\
           & = 0.1+0.4+0.3\nonumber               \\
    P(X>1) & = 0.8 \label{eq:problema3_px1}
\end{align}

Usando las ecuaciones \ref{eq:problema3_pxix1} y \ref{eq:problema3_px1} en la expansión de la ecuación \ref{eq:definition_ex2x1} se obtiene lo siguiente:

\begin{align*}
    E(X^2|X>1) & = \sum_i x_i^2 P(X=x_i | X >1 )                                                                                   \\
               & = 1^2 P(X=1|X>1) + 2^2 P(X=2|X>1)+3^2P(X=3|X>1) +4^2 P(X=1|X>1)                                                   \\
               & = 4\left( \frac{P(X=2)}{P(A)} \right)+ 9 \left(\frac{P(X=3)}{P(A)}\right) + \left( 16 \frac{P(X=4)}{P(A)} \right) \\
               & = \frac{1}{P(A)} \left(4 P(X=2) + 9 P(X=3)+ 16 P(X=4) \right)                                                     \\
               & = \frac{1}{0.8} \left(4(0.1)+9(0.4)+16(0.3) \right)                                                               \\
               & = \frac{1}{0.8}\left(8.8\right)                                                                                   \\
    E(X^2|X>1) & = 11
\end{align*}

Por lo tanto:
\begin{equation*}
    E(X^2|X>1)  = 11
\end{equation*}