\section*{Problema 5}

\textbf{Un fabricante de misiles pretende que la precisión de sus misiles de larga distancia de la marca Palomas de Paz es tal que la variable R que mide la distancia entre donde cayó un misil y su destino original (en km.), tiene la siguiente densidad sobre [0,1]:}

\begin{equation*}
    f_R(r) = 2(1-r)
\end{equation*}

\textbf{ER y Var(R). Calcula la probabilidad de que un misil caiga a menos de 100 metros de su destino original, lo que se considera como objetivo destruido. Si los vuelos de cada misil son independientes entre sí ¿ Cuántos misiles se tienen que lanzar en promedio para que el objetivo quede destruido?}

Calculando $ER$:

\begin{align*}
    ER & = \int_{-\infty}^\infty rf_R(r)dr                                 \\
       & =\int_{0}^1 rf_R(r)dr                                             \\
       & = \int_0^1 2(r-r^2)dr                                             \\
       & = 2 \int_0^1 (r-r^2)dr                                            \\
       & = 2 \left.\left( \frac{r^2}{2} - \frac{r^3}{3} \right)\right|_0^1 \\
       & = 2 \left(\frac{1}{2}- \frac{1}{3} \right)                        \\
       & = 2 \left(\frac{1}{6}\right)                                      \\
    ER & = \frac{1}{3}
\end{align*}

Calculando $Var(R)$, como ya se tiene cuanto vale $ER$, entonces se calculara $E(R^2)$.

\begin{align*}
    E(R^2) & =\int_{-\infty}^\infty r^2f_R(r)dr                             \\
           & = \int_{0}^1 r^2f_R(r)dr                                       \\
           & = \int_0^1 2(r^2-r^3) dr                                       \\
           & = 2\int_0^1 (r^2-r^3) dr                                       \\
           & = 2\left.\left( \frac{r^3}{3}- \frac{r^4}{4}\right)\right|_0^1 \\
           & = 2\left(\frac{1}{3} -\frac{1}{4} \right)                      \\
           & = 2\left( \frac{1}{12}\right)                                  \\
    E(R^2) & = \frac{1}{6}
\end{align*}

Entonces:

\begin{align*}
    Var(R) & = E(R^2) -(ER)^2                           \\
           & = \frac{1}{6} - \left(\frac{1}{3}\right)^2 \\
           & = \frac{1}{6} - \frac{1}{9}                \\
    Var(R) & = \frac{1}{18}
\end{align*}

Sea A un evento en el cual un misil impacte a menos de 100 metros de su destino original.

\begin{equation*}
    A = \{ \omega \in R: R(\omega)\leq 0.1 \}
\end{equation*}

entonces

\begin{align*}
    P(A) & = \int_0^{0.1} f_R(r) dr                               \\
         & = \int_0^{0.1} 2(1-r) dr                               \\
         & = 2 \int_0^{0.1} (1-r) dr                              \\
         & = 2  \left.\left(r-\frac{r^2}{2}\right)\right|_0^{0.1} \\
         & = 2 \left(0.1-\frac{(0.1)^2}{2}\right)                 \\
         & = 2 (0.095)                                            \\
    P(A) & = 0.19
\end{align*}

Como el vuelo de cada misil es independiente, entoces R puede ser representado con una distribución geometrica ($R\sim Geo(p)$). Entonces, el valor esperado de  la cantidad de misiles que te tienen que lanzar para que el objetivo quede destruido es:

\begin{align*}
    E(A) & = \frac{1-P(A)}{P(A)} \\
         & = \frac{1-0.19}{0.19} \\
    E(A) & = 4.2631
\end{align*}