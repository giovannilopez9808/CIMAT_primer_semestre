\item \textbf{Eliges al azar un número entre 1 y 1200. Calcula la probabilidad de que sea divisible por 4 o por 6.}

La cantidad de números que son divisibles por 4 en el rango 1-1200 es 300. Para el mismo rango, la cantidad de números divisibles por 6 es 200. Existen números que son divisibles por 4 y 6, por ejemplo el número 12, el total de estos números es 100. Condensando esta información obtenemos que:
\begin{equation*}
    P(A)= \frac{300}{1200} \qquad P(B)= \frac{200}{1200}
\end{equation*}
donde $P(A)$ es la probabilidad de que el número elegido sea divisible por 4, $P(B)$ es la probabilidad de que el número elegido sea divisible por 6. Entonces, la probabilidad de obtener un número que sea divisible por 4 y 6 es:
\begin{equation*}
    P(A\cap B)= \frac{100}{1200}
\end{equation*}
Por lo tanto, la probabilidad de obtener un número que sea divisible por 4 o 6 es:
\begin{align*}
    P(A\cup B) & = P(A)+P(B)-P(A\cap B)                                   \\
               & = \frac{300}{1200} + \frac{200}{1200} - \frac{100}{1200} \\
               & = \frac{400}{1200}                                       \\
               &                                                          \\
    P(A\cup B) & = \frac{1}{3}
\end{align*}

