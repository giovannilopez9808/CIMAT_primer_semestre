\item \textbf{Lee en las notas (capítulo 1 del pdf en Moodle) lo de los dados no transitivos. Verifica los cálculos.}

Las caras de los dados son los siguientes:
\begin{itemize}
    \item dado A: 4, 4, 4, 4, 0, 0
    \item dado B: 3, 3, 3, 3, 3, 3
    \item dado C: 6, 6, 2, 2, 2, 2
    \item dado D: 5, 5, 5, 1, 1, 1
\end{itemize}
Para comprobar que $P(A>B)=\frac{2}{3}$, se tiene que las combinaciones donde A>B es 24, esto es porque podemos obtener cuatro veces el número cuatro en el dado A, y en el dado B podemos solamente obtener un 3. El total de combinaciones es 36, por lo tanto:
\begin{align*}
    P(A>B) & = \frac{24}{36} \\
    P(A>B) & = \frac{2}{3}
\end{align*}
Comprobando que $P(B>C)=\frac{2}{3}$, se tiene que las combinaciones donde B>C es 24. El número 6 en el dado C se repite dos veces y el 2 se repite cuatro veces, en el dado B contiene unicamente números 3. El total de combinaciones es 36, por lo tanto:
\begin{align*}
    P(B>C) & = \frac{24}{36} \\
    P(B>C) & = \frac{2}{3}
\end{align*}
De igual manera, para comprobar $P(C>D)=\frac{2}{3}$, se tiene que el número 6 se repite dos veces en el dado C, como el seis es mayor a cualquier número en el dado D, entonces se tendrian 12 combinaciones donde C es mayor. Por otro lado el número dos se repite cuatro veces y el dado D contiene tres veces el número 1, entonces obtenemos 12 combinaciones donde C>D. Dando como resultado que el total de combinaciones donde C>D es 24, por lo tanto:
\begin{align*}
    P(C>D) & = \frac{24}{36} \\
    P(C>D) & = \frac{2}{3}
\end{align*}
Comprobando que $P(A>D)=\frac{1}{3}$, el unico caso cuando A>D es cuando obtenemos un 4 en el dado A y un 1 en el dado D, dando asi un total de 12 combinaciones, por lo tanto:
\begin{align*}
    P(A>D) & = \frac{12}{36} \\
    P(A>D) & = \frac{1}{3}
\end{align*}
comprobando así, todos los datos que se muestran en el primer capitulo de las notas.