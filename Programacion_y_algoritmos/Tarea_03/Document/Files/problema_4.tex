\section*{Problema 4}
\textbf{Escribe una función que reciba dos arreglos de enteros ordenados en forma no decreciente, y combine dichos arreglos en un solo arreglo también ordenado en forma no decreciente.}

Se creo un programa el cual recibe dos listas de numeros de $n_1$ y $n_2$, donde $n_1$ y $n_2$ pueden tomar cualquier valor entero positivo. Estas listas pueden estar desordenadas, ya que internamente se ordenan de menor a mayor cada lista. Este paso se realiza siguiendo el algoritmo de \textit{quick sort} el cual se encuentra en el archivo \textcolor{title}{quick\_sort.h}. La acción de unir a estas dos listas se encuentra el el archivo \textcolor{title}{merge.h}. El algoritmo que se empleo es el siguiente:

\begin{lstlisting}[language=python]
    i = 0
    j = 0
    k = 0
    while i<n1 or j<n2:
        if data1[i] < data2[j]:
            if i<n1:
                merge[k] = data1[i]
                i += 1
            else:
                merge[k] = data2[j]
                j += 1
        else:
            if i<n1:
                merge[k] = data2[j]
                j += 1
            else:
                merge[k] = data1[i]
                i += 1
        k += 1
\end{lstlisting}

Las variables inicializadas en las lineas 1 al 3 son las encargadas del control en las posiciones de los datos 1, datos 2 y su union (merge). El ciclo iniciado en la linea 4 se detendra cuando las posiciones de cada arreglo hayan llegado a su maximo dando por resultado el arreglo merge con todos los datos. El if de la linea 5 decide que lista de datos será asignada, ya que se quiere un arreglo creciente. Los if de las lineas 6 y 13 realizan el control de no acceder a un valor de una lista dos veces o a una memoria invalida, ya que, puede darse el caso que un arreglo haya sido escrito totalmente en merge y su ultimo elemento sea menor a los elementos restantes del otro arreglo que aun no han sido escritos.\\

El programa se encuentra en la carpeta \textcolor{citecolor}{Problema\_4}.
El programa tiene la opción de recibir valores de un usuario o crear datos de forma aleatoria. El output esperado del programa con los valores $data_1=\{-1,4,6,1\}$ y $data_2=\{30,1,4,2,5,7\}$ es el siguiente:

\begin{lstlisting}[language=bash]
    Deseas usar el programa con los datos de prueba?(Y/n): n
    Escribe el tamano de los datos 1: 4
    Escribe el tamano de los datos 2: 6

    Datos 1:
    Escribe el numero 1 de 4: -1
    Escribe el numero 2 de 4: 4
    Escribe el numero 3 de 4: 6
    Escribe el numero 4 de 4: 1

    Datos 2:
    Escribe el numero 1 de 6: 30
    Escribe el numero 2 de 6: 1
    Escribe el numero 3 de 6: 4
    Escribe el numero 4 de 6: 2
    Escribe el numero 5 de 6: 5
    Escribe el numero 6 de 6: 7


    Numbers 1:
    -1 1 4 6 

    Numbers 2:
    1 2 4 5 7 30 

    Merge numbers:
    -1 1 1 2 4 4 5 6 7 30 
\end{lstlisting}

El output para los valores generados aleatoriamente es el siguiente:

\begin{lstlisting}[language=bash]
    Deseas usar el programa con los datos de prueba?(Y/n): y

    Numbers 1:
    -46 -25 -15 27 39 

    Numbers 2:
    -50 -43 -32 8 13 38 48 

    Merge numbers:
    -50 -46 -43 -32 -25 -15 8 13 27 38 39 48 

\end{lstlisting}

El comando para compilar el programa es el siguiente:
\begin{lstlisting}[language=bash]
    gcc -Wall -Wextra -Werror -pedantic -ansi -o main.out main.c -std=c11
\end{lstlisting}