\section*{Problema 3}

\textbf{Dado una lista de nombres (strings) de N personas (apellido paterno, apellido materno, Nombre(s)), escribir un programa que ordene los nombres alfabéticamente usando un arreglo de apuntadores. Los nombres pueden tener distintas longitudes; cuando un nombre sea prefijo de otro, considerar al nombre mas corto como menor. El ordenamiento debe ser a través de una función que reciba el arreglo de apuntadores.}

Se creo una estructura la cual contiene tres elementos para el apellido paterno, apellido materno y nombre(s). Cada linea del archivo de nombres será una persona diferente. El orden que se opto fue primero por apellido paterno, si se encuentra que son iguales pasa a ordenar comparando el apellido materno y por último el nombre de la persona. Se utilizo la función \textcolor{title}{sorted} creada en tareas anetriores. Se modifico para comparar nombres con la función \textcolor{title}{compare\_names} y \textcolor{title}{order\_names}. La función \textcolor{title}{order\_names} es la siguiente:

\begin{lstlisting}[style=CStyle]
    int order_names(char name1[], char name2[])
    {
        int i = 0;
        int compare;
        while (name1[i] != '\0' && name2[i] != '\0')
        {
            compare = name1[i] - name2[i];
            if (compare != 0)
            {
                return compare;
            }
            i = i + 1;
        }
        if (name1[i] == '\0' && name2[i] != '\0')
        {
            return -name2[i];
        }
        if (name2[i] == '\0' && name1[i] != '\0')
        {
            return name1[i];
        }
        return 0;
    }
\end{lstlisting}

Los parámetros name1 y name2 son introducidos a partir de un puntero hacia la estructura. En la linea 10 se devuelve el producto de la ``resta'' de caracteres de la linea 7. Si este valor es menor a 0 entonces name2 tiene un orden alfabético mayor que name1. Si son iguales entonces el valor de su resta es 0. En el caso en que un nombre o apellido sea prefijo de otro entonces el ciclo iniciado en la linea 5 terminará, ya que uno de los nombres habra llegado al final. Si name1 que llego a su final, entonces se devolverá el negativo de la i-esima letra de name2 (linea 16), si name2 llego a su final, entonces se devolverá la i-esima letra de name1 (linea 20).

Se uso el archivo \textcolor{citecolor}{names.txt} que contiene los siguientes nombres:

\begin{lstlisting}[language=bash]
    Venere Bookamer Art Art 
    Paprocki Kampa Lenna Lenna
    Paprocki Kampa Lenna Le
    Dar Biddy Josephine Josephine
    Darakjy Biddy Josephine Josephine
    Foller Gillaspie Donette Donette
    Butt Motley James James
    Morasca Harabedian Simona Simona
    Paprocki Kampa Lenna Lo
    Tollner Hixenbaugh Mitsue Mitsue
    Dilliard Oles Leota Leota
    Foller Gilla Donette Donette
    Paprocki Kampa Le
    Wieser Ankeny Sage
\end{lstlisting}

El programa se encuentra en la carpeta \textcolor{citecolor}{Problema\_3}. Para ejecutar el programa se uso el siguiente comando:

\begin{lstlisting}[language=bash]
    ./main.out names.txt
\end{lstlisting}

El cual produce el siguiente output:

\begin{lstlisting}
    ----------------------------------
    Nombres desordenados
    
    Venere Bookamer Art Art 
    Paprocki Kampa Lenna Lenna
    Paprocki Kampa Lenna Le
    Dar Biddy Josephine Josephine
    Darakjy Biddy Josephine Josephine
    Foller Gillaspie Donette Donette
    Butt Motley James James
    Morasca Harabedian Simona Simona
    Paprocki Kampa Lenna Lo
    Tollner Hixenbaugh Mitsue Mitsue
    Dilliard Oles Leota Leota
    Foller Gilla Donette Donette
    Paprocki Kampa Le
    Wieser Ankeny Sage
    
    
    ----------------------------------
    Nombres ordenados
    
    Butt Motley James James
    Dar Biddy Josephine Josephine
    Darakjy Biddy Josephine Josephine
    Dilliard Oles Leota Leota
    Foller Gilla Donette Donette
    Foller Gillaspie Donette Donette
    Morasca Harabedian Simona Simona
    Paprocki Kampa Le
    Paprocki Kampa Lenna Le
    Paprocki Kampa Lenna Lenna
    Paprocki Kampa Lenna Lo
    Tollner Hixenbaugh Mitsue Mitsue
    Venere Bookamer Art Art 
    Wieser Ankeny Sage
\end{lstlisting}

La compilación el programa se uso el siguiente comando:

\begin{lstlisting}[language=bash]
    gcc -Wall -Wextra -Werror -pedantic -ansi -o main.out main.c -std=c11
\end{lstlisting}