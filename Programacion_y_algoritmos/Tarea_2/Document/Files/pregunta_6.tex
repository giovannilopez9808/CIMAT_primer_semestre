\section*{Problema 6}
\textbf{Programa que convierta un número decimal a cualquier base.}

La estructura del código es la siguiente:
\begin{lstlisting}[language=python]
    basis = 16
    number = 2342
    number_c = ""
    while(number!=0):
        residue = number % basis
        number = int(number / basis)
        number_c += integer_to_ascii(residue)
    number_c = inverse_string(number_c)
    print(number_c)

    > output: 926
\end{lstlisting}

Para hacer una prueba de este algorimo usaremos la base 16 con el número 2342. El algoritmo desglosado se muestra en la tabla \ref{table:test_problem6}.
\begin{table}[H]
    \centering
    \begin{tabular}{cccc} \hline
        \textbf{Operación} & \textbf{Cociente} & \textbf{Residuo} & \textbf{ASCII(Residuo)} \\ \hline
        2342 / 16          & 146               & 6                & 6                       \\
        146 / 16           & 9                 & 2                & 2                       \\
        9 / 16             & 0                 & 9                & 9                       \\\hline
    \end{tabular}
    \caption{Prueba del algoritmo de conversion de base con el número 2342 y la base 16.}
    \label{table:test_problem6}
\end{table}
donde como resultado que el número 2342 en base 16 es 926.

El programa se encuentra en la carpeta \textcolor{citecolor}{Problema\_6}. La manera de ejecutar el programa es con la siguiente linea:
\begin{lstlisting}[language=bash]
    gcc -Wall -o main.out main.c -std=c11
\end{lstlisting}
