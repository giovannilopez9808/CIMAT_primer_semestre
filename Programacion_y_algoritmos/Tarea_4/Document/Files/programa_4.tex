\section*{Problema 4}

\textbf{Programa que encuentre los tres números mayores de un arreglo de enteros, especificando su posición (índice) original en el arreglo de entrada.}

El programa que genera 20 números aleatorios entre 0 y 100, este valor puede ser cambiado en la definición de la linea 5 del archivo \textcolor{citecolor}{main.c}. La posición inicial de cada número se establece cuando es generado. Esta posición es guardada en un arreglo de dimensión 20.

En seguida se utilizo el algoritmo de quick sort, el cual fue modificado para recibir el arreglo de posiciones originales de cada número. En la función \textcolor{title}{reduce\_sort} del archivo \textcolor{citecolor}{quick\_sort.} es la siguiente:

\begin{lstlisting}[style=CStyle]
    int reduce_sort(int numbers[], int positions[], int low, int high)
    {
        int pivot = numbers[high];
        int i = low;
        for (int j = low; j <= high - 1; j++)
        {
            if (numbers[j] >= pivot)
            {
                swap(&numbers[i], &numbers[j]);
                swap(&positions[i], &positions[j]);
                i++;
            }
        }
        swap(&numbers[i], &numbers[high]);
        swap(&positions[i], &positions[high]);
        return i;
    }
\end{lstlisting}

Se añadio la linea 10 y 15 para seguir la posición de cada valor que es ordenado. El algoritmo ordena de mayor a menor los valores contenidos en la variable numbers. El programa se encuentra en la carpeta \textcolor{citecolor}{Problema\_4}. Uno de los output del programa es el siguiente:

\begin{lstlisting}[language=bash]
    ----------------------------------
    Impresion de los 20 primeros datos

    Posicion	Valor
    0		995
    1		225
    2		52
    3		410
    4		653
    5		652
    6		40
    7		411
    8		954
    9		38
    10		564
    11		416
    12		714
    13		416
    14		902
    15		851
    16		169
    17		243
    18		707
    19		393

    ----------------------------------
    Impresion de los 3 primeros datos

    Posicion	Valor
    0		995
    8		954
    14		902
\end{lstlisting}


La compilación del programa fue realizadacon el siguiente comando.

\begin{lstlisting}[language=bash]
    gcc -Wall -Wextra -Werror -pedantic -ansi -o main.out main.c -std=c11
\end{lstlisting}