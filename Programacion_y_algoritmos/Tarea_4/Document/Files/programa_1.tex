\section*{Problema 1}
\textbf{Escribir un programa que genere una caminata aleatoria en una matri de 10x10. El arreglo debe contener inicialmente puntos ``.", y debe recorrers basado en el residuo de un número aleatoria (usar srand y rand()) cuyos resultado puede ser \{0 (arriba), 1 (abajo), 2 (izq), 3 (der)\}, que indican la dirección a moverse. A) Verificar que el movimiento no se salga del arreglo de la matriz, y B) No se pued visitar el mismo lugar más de una vez. Si alguna de estas condiciones intentar moverse hacia otra dirección; si todas las posiciones están ocupadas, finalizar el programa e imprimi el resultado.}

Para este problema se definidio como una constante global el tamaño de la matriz y el maximo de letras posibles. Los valores son 10 y 26 respectivamente. Esto porque existen 26 letras en el código ascii ignorando a la letra ñ. El estado inicial de la matriz contiene unicamente puntos, esto puede verse en la figura \ref{fig:initial_state_problem1}. El programa inicia de dos maneras diferentes, un usuario introduce un punto inicial donde comenzará la caminata o este es escogido aleatoriamente.

\begin{figure}[H]
    \centering
    \includegraphics[width=4cm]{Graphics/initial_problem1.png}
    \caption{Estado inicial de la matriz}
    \label{fig:initial_state_problem1}
\end{figure}

Una casilla se considera disponible si su valor es igual a ``.". Antes de obtener el siguiente paso de cada iteracion el programa verifica si sus vecinos estan disponibles. Si existe al menos una casilla disponible entonces el programa sigue ejecutandose (figura \ref{fig:example_walk}), si no el programa termina ya que no podrá seguir la caminata (figura \ref{fig:example_block}). El siguinte paso de la iteración se obtiene con la función \textit{rand()}. Se calcula el modulo del número obtenido con 4. Con esto nos aseguramos que el número se encuentre en el conjunto $\{0, 1, 2 ,3\}$.
\begin{center}
    \begin{minipage}{0.48\linewidth}
        \begin{figure}[H]
            \centering
            \includegraphics[width=4cm,height=4cm]{Graphics/example_2_1_problem1.png}
            \includegraphics[width=4cm,height=4cm]{Graphics/example_2_2_problem1.png}
            \caption{Ejemplo de caminata.}
            \label{fig:example_walk}
        \end{figure}
    \end{minipage}
    \begin{minipage}{0.48\linewidth}
        \begin{figure}[H]
            \centering
            \includegraphics[width=4cm,height=4cm]{Graphics/example_3_1_problem1.png}
            \includegraphics[width=4cm,height=4cm]{Graphics/example_3_2_problem1.png}
            \caption{Ejemplo de termino del programa.}
            \label{fig:example_block}
        \end{figure}
    \end{minipage}
\end{center}

Cada paso debe estar contenido dentro del arreglo de 10x10, si el paso sobrepasa la barrera este es rechazado y se calcula otro diferente. Esto es realizado con la función \textcolor{citecolor}{is\_in\_the\_box} con el siguiente algoritmo:

\begin{lstlisting}[style=CStyle]
    int is_in_the_box(int pos[])
    {
    for (int i = 0; i < 2; i++)
    {
        if ((pos[i] < 0) || (pos[i] >= size))
        {
            return 0;
        }
    }
    return 1;
    }
\end{lstlisting}
Si esta dentro de los límites la función devuelve 1, si esta fuera 0. En la figura \ref{fig:example_final_problema1} se muestran diferentes resultados obtenidos por el programa:

\begin{figure}[H]
    \centering
    \includegraphics[width=4cm,height=4cm]{Graphics/example_4_1_problem1.png}
    \includegraphics[width=4cm,height=4cm]{Graphics/example_4_2_problem1.png}
    \includegraphics[width=4cm,height=4cm]{Graphics/example_4_3_problem1.png}
    \includegraphics[width=4cm,height=4cm]{Graphics/example_4_4_problem1.png}
    \caption{Diferentes output del programa}
    \label{fig:example_final_problema1}
\end{figure}

El programa se encuentra en la carpeta \textcolor{citecolor}{Problema\_1}. Para compilar el programa se uso el siguiente comando:
\begin{lstlisting}[language=bash]
    gcc -Wall -Wextra -Werror -pedantic -ansi -o main.out main.c -std=c11
\end{lstlisting}
