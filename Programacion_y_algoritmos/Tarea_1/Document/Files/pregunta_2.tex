\item \textbf{Describir detalladamente los paradigmas de computación principales y su relacion. Dar ejemplos de cada paradigma.}

Los paradigmas de programación indica un método de realizar cómputos ,la estructura y organización de las tareas que se deben realizar en un programa. Los paradigmas principales están basados en diferentes modelos de cómputo, por lo tanto afectan a las construcciones más básicas de un programa. Los paradigmas principales son:
\begin{itemize}
      \item Imperativo.
      \item Declarativo.
      \item Lógico.
      \item Funcional.
      \item Orientado a objetos.
\end{itemize}
\subsubsection*{Imperativo}
El paradigma imperativo consiste en en una serie de sentencias ejecutadas según un control de flujo explícito que modifican el estado del programa.
\subsubsection*{Declarativo}
El paradigma declarativo se describe en términos de su respuesta, pero no se define explícitammente qué propiedades se espera que presente el resultado\cite{Daintith_2008}.Contiene un control del flujo que esta asociado a una composición de funciones, recursividad y/o técnicas de reescritura y unificación.
\subsubsection*{Lógico}
El paradigma lógico consiste en declarar una serie de hechos conocidos y relacionados. Se debe establecer una serie de propiedades para que en la ejecución se base en un mecanismo deductivo.
\cite{satori_2021,rodriguez_2003}
\subsubsection*{Funcional}
En las matemáticas se establecio el concepto de \textit{función}, las cuales establecen una relación entre los parámetros y el resultado. Un programa consiste en la definición de una o más funciones. Para la ejecución del programa se establecen parámetros y el programa realiza las operaciones necesarias para obtener el resultado.\cite{Fokker_1996}
\subsubsection*{Orientado a objetos}
Un lenguaje de programación orientado a objetos puede contener el paradigma imperativo, funcional o lógico. Lo que caracteriza este estilo es la forma de manejar la información basada en los conceptos clase, objeto y herencia.
\begin{itemize}
      \item Clase

            Tipo de dato con propiedades determinadas con funciones.
      \item Objeto

            Entidad de una clase con una serie de parámetros capaces de interactuar con otros objetos.
      \item Herencia

            Propiedad por la que es posible construir una nueva clase a partir de una ya existente.
\end{itemize}