\item \textbf{Describir detalladamente los paradigmas de computación principales y su relacion. Dar ejemplos de cada paradigma.}

Los paradigmas de programación indica un método de realizar cómputos ,la estructura y organización de las tareas que se deben realizar en un programa. Los paradigmas principales están basados en diferentes modelos de cómputo, por lo tanto afectan a las construcciones más básicas de un programa. Los paradigmas principales son:
\begin{itemize}
    \item Imperativo.
    \item Declarativo.
    \item Lógico.
    \item Funcional.
    \item Orientado a objetos.
\end{itemize}
\subsubsection*{Imperativo}
EL paradigma imperativo consiste en en una serie de sentencias ejecutadas según un control de flujo explícito que modifican el estado del programa.
\subsubsection*{Declarativo}
El paradigma declarativo se describe en términos de su respuesta. Contiene un control del flujo que esta asociado a una composición de funciones, recursividad y/o técnicas de reescritura y unificación.
\subsubsection*{Lógico}
\cite{satori_2021}
\subsubsection*{Funcional}
\subsubsection*{Orientado a objetos}