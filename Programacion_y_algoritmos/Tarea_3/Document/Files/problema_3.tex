\section*{Problema 3}
\textbf{Escribe una función que reciba un arreglo de enteros y que calcule el producto acumulado para cada entrada usando la ecuación \ref{eq:product_problem3}. Se espera que la complejidad del algoritmo sea lineal O(n). Se puede asumir que la longitud máxima del arreglo será 50.}
\begin{equation}
    a[i] = \prod_{j\neq i} a[j]
    \label{eq:product_problem3}
\end{equation}

El algoritmo que se ideo es el siguiente:

\begin{lstlisting}[language=python]
    all_product=1
    size=len(data)
    product = []
    for i in range(size):
        all_product = all_product * data[i]
    for i in range(size):
        product.push(all_product / data[i])
\end{lstlisting}

En el ciclo de la linea 4 se calcula el producto de todos los números contenidos en la lista de datos. En el ciclo de la linea 6 se guardan los productos acumulados siguiendo la ecuación \ref{eq:product_problem3}. Se esta manera se obtiene con una complejidad de O(n). El programa se encuentra en la carpeta \textcolor{citecolor}{Problema\_3}. El programa acepta valores de entrada especificando cuantos valores se introducira. Un ejemplo de esto es con la siguiente lista de valores: $data=\{3,5,3,2\}$, el output que da el programa es el siguiente:

\begin{lstlisting}[language=bash]
    Deseas usar el programa con los datos de prueba?(Y/n): n

    Escribe el tamano de los datos: 4
    Escribe el numero 1 de 4: 3
    Escribe el numero 2 de 4: 5
    Escribe el numero 3 de 4: 3
    Escribe el numero 4 de 4: 2

    ------------------------------------------------

    Numbers	Product
    3		30
    5		18
    3		30
    2		45
\end{lstlisting}

Los datos de ejemplo son creados aleatoriamente. El output de uno de ellos es el siguiente:

\begin{lstlisting}
    Deseas usar el programa con los datos de prueba?(Y/n): y

    ------------------------------------------------

    Numbers	Product
    8		87210
    17		41040
    2		348840
    19		36720
    9		77520
    15		46512
\end{lstlisting}

El comando para compilar el programa es el siguiente:

\begin{lstlisting}[language=bash]
    gcc -Wall -Wextra -Werror -pedantic -ansi -o main.out main.c -std=c11
\end{lstlisting}