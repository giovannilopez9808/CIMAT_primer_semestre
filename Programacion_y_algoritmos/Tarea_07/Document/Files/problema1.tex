\section{Análisis de la complejidad de Quicksort}

El algoritmo de Quicksort puede resumirse en lo siguiente:

\begin{algorithm}[H]
    \caption{Quicksort}
    \label{alg:quicksort}
    \KwInput{data}
    \KwOutput{data}
    \If{start$<$last}
    {
        number\_data $\gets$ reduce\_data(data,start,last)\\
        sort(data,start,number\_data)\\
        sort(data,number\_data+1,last)\\

    }
    \SetKwFunction{FMain}{reduce\_data}
    \SetKwProg{Fn}{Function}{:}{}
    \Fn{\FMain{$F$}}{
    pivot $\gets$ data[last]\\
    i $\gets$ start\\
    \For{j=start,last-1}
    {\If{data[j]$>$pivot}
    {swap(data[j],data[i])\\
    i $\gets$ i+1}
    }
    return i\\
    }
\end{algorithm}

Con este algoritmo podemos realizar dos análsis, esto es para el peor caso y mejor caso. En el algoritmo \ref{alg:quicksort} se ve que es recursivo. En la linea 3 y 4 se puede ser que se da como argumento la posición inicial y final de los datos a ordenar. El algortimo acaba cuando las dos funciones tienen el mismo en valor en la posición inicial y final.

\subsection{Peor caso}

El peor caso posible es que el pivote escogido siempre sea el número más grande o pequeño del arreglo. Esto es porque provocaria que uno de los bloques sea de tamaño n y otro de 1. Esto sería:

\begin{align*}
    T(n) & = T(n-1) + n              \\
         & = T(n-2)+ (n-1) + n       \\
         & = T(n-3)+(n-2)+ (n-1) + n \\
         & \qquad \vdots             \\
         & =1 +2 +3 + \cdots +n      \\
         & =\frac{n(n+1)}{2}
\end{align*}

Entonces, la complejidad para el peor caso es $O(n^2)$

\subsection{Mejor caso}

El mejor caso posible es que el pivote escogido sea el número sea la mediana del arreglo. Esto provocaria que los dos bloques tengan un tamaño de $\frac{n}{2}$. Entonces:

\begin{align*}
    T(n) & = 2T\left(\frac{n}{2}\right) + n                           \\
         & = 2\left(2T\left(\frac{n}{2}\right)+\frac{n}{2}\right) + n \\
         & = 4T\left(\frac{n}{4}\right)+2n                            \\
         & = 2^k T\left(\frac{n}{2^k}\right) + kn                     \\
         & =nT(1)+ log(n)(n) \qquad \text{si } n=2^k                  \\
         & = n+ nlog(n)
\end{align*}

Entonces, la complejidad para el mejor caso es $O(nlog(n))$.