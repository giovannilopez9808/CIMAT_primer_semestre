\section{Archivo de entrada \label{sec:read_file}}

El archivo que contiene los datos iniciales del programa se dará al momento de iniciarse el programa. Este archivo debe contener como cabecera los siguientes titulos: \textit{Calificacion, Edad, Grupo, Turno} y \textit{Nombre}.

Los datos que contiene el archivo inicialmente se muestran en la tabla \ref{table:initial_data}.

\begin{table}[H]
    \centering
    \begin{tabular}{ccccl} \hline
        Calificacion & Edad & Grupo & Turno & Nombre             \\ \hline
        A+           & 9    & C     & V     & Valeria Quirarte   \\
        A-           & 12   & A     & V     & Valentina Quirarte \\
        B            & 10   & D     & M     & Luis Aldama        \\
        B-           & 14   & A     & V     & Marco Lopez        \\
        A            & 15   & A     & V     & Mayra Medellin     \\
        C            & 13   & E     & M     & Pedro Arturo       \\ \hline
    \end{tabular}
    \caption{Datos iniciales del archivo \file{data.txt}.}
    \label{table:initial_data}
\end{table}

\subsection{Lectura de datos}

Al inicio del programa se obtiene el número de personas (filas) que contiene el archivo. Esta cantidad es obtenida en la función \script{obtain\_size}. Con esta cantidad es iniciado el arreglo de estructuras \script{students}. Al tener ya resevada la memoria para los datos de la tabla \ref{table:initial_data}, se realiza la lectura de datos. Este mecanismo es realizado en la función \script{obtain\_information}.