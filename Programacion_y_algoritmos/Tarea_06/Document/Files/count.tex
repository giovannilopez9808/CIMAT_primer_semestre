\section{Contabilización}

El menú contiene diferentes modos para contabilizar la información de los estudiantes. Los criterios creados son los siguientes:

\begin{itemize}
    \item Cobtabilización por grupo
    \item Contabilización por turno
\end{itemize}

Los dos criterios reciben un arreglo el cual llevara el conteo de cada tipo. Al inicio de cada criterio se inicializa en ceros todos los elementos del arreglo, ya que no se conoce si se realizó algún cambio en los datos. Los programas de esta sección se encuentran contenidos en el archivo \file{count.h}.

\subsection{Contabilización por grupo}

Los grupos existentes son los siguientes: A, B, C, D , E y F. Entonces el arreglo que recibe debe ser de dimensión 6. Al finalizar el conteo de este arreglo se ejecuta una función para imprimir en pantalla el número de alumnos en cada grupo.

\subsection{Contabilización por turno}

Los turnos existentes son matutino (M) y vespertino (V). Entonces el arreglo que recibe debe ser de dimensión 2. Al finalizar el conteo de este arreglo se ejecuta una función para imprimir en pantalla el número de alumnos en cada turno.